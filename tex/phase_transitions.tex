\section{Todo notes}

Equations of state
\begin{itemize}
    \item Bag model \cite[eq. 7.33]{lecture_notes}
    \item Assumptions on speed of sound and constant parameters
    \item Giese's paper \cite{giese_2020}
    \item Enqvist, old paper \cite{enqvist_nucleation_1992}
\end{itemize}

Employment?
- "things that need to be done"
-> suitable for public dissemination
- write some short description,
    - ArXiv paper

2 papers by Giese, second one is more complete
Paper 1 \cite{giese_2020}
Paper 2 \cite{giese_2021}

TODO: understand these
\begin{itemize}
    \item Relativistic hydrodynamics from the review article (aka. lecture notes) \cite{lecture_notes}
    \item Reviews from other people (not just Hindmarsh)
    \item Article by Mazumdar \& White, especially section 7: relativistic combustion \cite{mazumdar_review_2019}
    \item The sound shell model paper
    \item Really understand what's going on! Compute with pen \& paper etc.
    \item Relativistic hydrodynamics
    \begin{itemize}
        \item The book by Luciano Rezzolla \& Olindo Zanotti \cite{rezzolla_relativistic_2013}
        \item Read the part I (some 300 pages at first)
        \item Contains more than necessary for the thesis
    \end{itemize}
    \item The phase transition section should have plenty of content before Christmas
    \item Enqvist's paper
    \begin{itemize}
        \item Good background: connects the equations of state to particle physics
        \item -> What kinds of equations of state would a particle theory produce
        \item Assumes ultrarelativisticity, which is no longer necessary due to the improvement of computational tools
    \end{itemize}
    \item Ultimately the equation of state can be derived from the "Thermal effective potential".
    \item Chemical potential is zero, as there is no conserved particle number.
    \item J-function, fB, lecture eq. 2.19 etc.
    \begin{itemize}
        \item These will be our first equation of state.
    \end{itemize}
\end{itemize}

"My job is to give the GWs the right energy-momentum tensor"

\section{Basics}
Electroweak symmetry breaking at $10^{-11} s$, $100 -- 1000 GeV$.
First-order phase transition
-> bubbles just below the critical temperature
-> gravitational waves

Standard model has a crossover, but many extensions first-order.

Matter-antimatter asymmetry -> net baryon number
(But isn't B an accidental symmetry, whereas B-L is a fundamental one?
How does this correspond to the neutrino content of the universe?)
\cite{lecture_notes}

The gravitational wave spectrum can be calculated from a few thermodynamic properties of ultrarelativistic matter, which can be computed from quantum field theory.
These parameters can be measured by LISA.

\section{Relativistic hydrodynamics}
Ultrarelativistic particles
Non-constant particle number

Statistical mechanics \cite{huang_statistical_1987}
\cite{schroeder_thermal_2000}
\cite[ch. 4]{lecture_notes}
\todo{Is it OK to have references on the level of individual chapters and equations?}

Let us start from the very basics of thermal physics.
The
\href{https://en.wikipedia.org/wiki/Clausius\%E2\%80\%93Clapeyron_relation}{Clausius-Clapeyron relation}
\todo{I have added some hyperlinks to Wikipedia to make further reading easier for the reader. Is this OK?}
\cite[eq. 5.47, 5.48]{schroeder_thermal_2000}
\begin{equation}
\frac{dp}{dT} = \frac{L}{T \Delta V} = \frac{\Delta S}{\Delta V},
\end{equation}
when taken to the differential limit, gives the entropy density
\cite[p. 23]{lecture_notes}
\begin{equation}
s \equiv \frac{dp}{dT}.
\end{equation}
This can be inserted in the thermodynamic identity
\cite[eq. 3.46]{schroeder_thermal_2000}
\begin{equation}
dU = TdS - pdV
\end{equation}
to get the energy density
\begin{equation}
e \equiv T \frac{\partial p}{\partial T} - p.
\end{equation}
This in turn can be inserted to the definition of enthalpy
\cite[eq. 1.51]{schroeder_thermal_2000}
\begin{equation}
H \equiv E + pV
\end{equation}
to get the enthalpy density
\begin{align}
w
&\equiv T \frac{\partial p}{\partial T} \\
&= e+p \\
&= Ts.
\end{align}
Additionally we need to define the entropy current as
\cite[p. 23]{lecture_notes}
\begin{equation}
S^\mu \equiv su^\mu
\end{equation}
It should be noted that the
\href{https://en.wikipedia.org/wiki/Einstein_notation}{Einstein notation} will be used for the indices throughout the thesis.
% The Einstein notation is absolute basics for theoretical physicists, but unfortunately the study programme of many of my applied/technical physicists friends doesn't involve it, so I'm mentioning it explicitly here so that they can google it.

\href{https://en.wikipedia.org/wiki/Stress\%E2\%80\%93energy\_tensor\#Stress\%E2\%80\%93energy\_of\_a\_fluid\_in\_equilibrium}{The energy-momentum tensor of a perfect fluid in thermodynamic equilibrium}
is of the form
\todo{The specifics of the energy-momentum tensor will vary by the model,
so highlight how this affects the rest of the equations. Todo: derive this from the fluid equations.}
\cites[eq. 5.11, 5.23]{lecture_notes}[eq. 3]{giese_2020}[eq. 4]{giese_2021}
\begin{align}
T^{\mu \nu}_f
&= (e+p) u^\mu u^\nu + p g^{\mu \nu}
\label{eq:ep_tensor} \\
&= w u^\mu u^\nu + p g^{\mu \nu}
\end{align}


\section{Field-fluid system}
Based on \cite{moore_pt_1995}.

In the electroweak phase transition the fluid is coupled to a scalar field with the energy-momentum tensor
\cites[eq. 2.9]{hindmarsh_gw_pt_2019}
\begin{equation}
T_\phi^{\mu \nu}
= \partial^\mu \partial^\nu \phi
- g^{\mu \nu} \left(\frac{1}{2} (\partial \phi)^2 + V_0 (\phi) \right)
\end{equation}

The total energy-momentum is conserved,
\cite[eq. 5.17]{lecture_notes}
\begin{equation}
\partial_\mu (T_f^{\mu \nu} + T_\phi^{\mu \nu} = 0.
\label{eq:ep_conservation}
\end{equation}
This total energy-momentum tensor is conserved.


\section{Relativistic combustion}
This chapter of the lecture notes is largely based on
\cites{hindmarsh_gw_pt_2019}{espinosa_energy_2010}

The conservation of the energy-momentum of equation \ref{eq:ep_tensor} results in the bubble wall junction conditions
\cites[eq. 7.22]{lecture_notes}[eq. B.2-3]{hindmarsh_gw_pt_2019}
\begin{align}
w_- \tilde{\gamma}_-^2 \tilde{v}_- &= w_+ \tilde{\gamma}_+^2 \tilde{v}_+, \\
w_- \tilde{\gamma}_-^2 \tilde{v}_-^2 + p_- &= w_+ \tilde{\gamma}_+^2 \tilde{v}_+^2 + p_+
\end{align}
The first one i
\todo{Derive the first equation}
The second one is derived by multiplying the conservation with $g^{\mu \nu}$. 

The bubble wall junction conditions can be rearranged as
\cites[eq. 7.32]{lecture_notes}[eq. 6-7]{giese_2020}
\begin{align}
\tilde{v_+} \tilde{v_-} = \frac{p_+ - p_-}{e_+ - e_-} \\
\frac{\tilde{v}_+}{\tilde{v}_-} = \frac{e_- + p_+}{e_+ + p_-}
\end{align}
And simplified to
\begin{align}
\frac{v_+}{v_-} &= \frac{1 - \frac{\Delta e}{w_+}}{1 - \frac{\Delta p}{w+}} \\
v_+ v_- &= \frac{\frac{\Delta p}{w_+}}{\frac{\Delta e}{w_+}}
\end{align}

Speed of sound
\cites[eq. 13]{giese_2020}[eq. 3]{giese_2021}
\begin{align}
c_s^2
&\equiv \frac{dp}{de} \\
&= \frac{dp/dT}{de/dT}
\end{align}
For a relativistic plasma $c_s^2=\frac{1}{3}$.

The trace of the energy-momentum tensor of equation \ref{eq:ep_tensor} is
\begin{align}
\text{tr} (T^{\mu \nu})
&= T^\mu_mu \\
&= g_{\mu \nu} T^{\mu \nu} \\
&=3p - e
\end{align}
We can quantify the deviation from this with the trace anomaly
\cites[eq. 7.24]{lecture_notes}[eq. 28]{giese_2020}
\begin{equation}
\theta = \frac{1}{4}(e-3p).
\end{equation}
\todo{Why is Hindmarsh using a 1/4 here, but Giese is not?}

Difference of trace anomaly
\begin{align}
\Delta \theta
&= \theta_+ - \theta_- \\
&= \theta_s - \theta_b
\end{align}

Transition strength
\begin{equation}
\alpha_+ = \frac{4 \Delta \theta}{3 w_+}
\end{equation}

Enthalpy ratio
\begin{equation}
r = \frac{w_+}{w_-}
\end{equation}

We can obtain useful continuity equations from the conservation of energy-momentum.
We start by defining two vectors, the fluid 4-velocity
\begin{equation}
u^\mu = \gamma(-1, \overrightarrow{v})^\mu
\end{equation}
and its orthonormal vector
\begin{equation}
\bar{u}^\mu = \gamma(-v, \frac{\overrightarrow{v}}{v})^\mu.
\end{equation}
These form an orthonormal basis.
It should be noted, that for orthonormal vectors
\begin{align}
% Note that the lecture notes use ubar here.
u_\mu u^\mu &= -1, \\
u_\mu \bar{u}^\mu &= 0.
\end{align}
By projecting the conservation equation to this basis we get
\begin{align}
0 = u_\mu \partial_\nu T^{\mu \nu} = -\partial_\mu (w u^\mu) + u^\mu \partial_\mu p \\
0 = \bar{u} \partial_\nu T^{\mu \nu} = w \bar{u}^\nu u^\mu \partial_\mu u_\nu + \bar{u}^\mu \partial_\mu p
\end{align}

Dimensionless coordinate
\begin{equation}
\xi = \frac{r}{t}
\end{equation}

Hydrodynamic equations, aka. continuity equations
\todo{Derive these}
\cites[eq. 7.30-7.31]{lecture_notes}[eq. 5]{giese_2021}
\begin{align}
\frac{dv}{d\chi} &= \frac{2v(1-v^2)}{\chi(1-v\chi)} \left( \frac{\mu(\chi,v)^2}{c_s^2} - 1 \right)^{-1} \\
\frac{dw}{d\chi} &= w \left( 1 + \frac{1}{c_s^2} \right) \gamma^2 \mu(\chi,v) \frac{dv}{d\chi}
\end{align}


\section{The bag model}
The bag model is the simplest model for the equation of state.
\todo{Find references to confirm this.}

In general, the speed of sound is temperature-dependent.
However, if both phases are assumed to be ultrarelativistic, the speed of sound simplifies to $c_s^2 = \frac{dp}{de} = 1/3$.
This is satisfied by the "bag" model, where
\cites[eq. 7.33]{lecture_notes}[eq. 8-9]{giese_2020}
\begin{align}
p_s &= a_s T^4 - V_s
\label{eq:bag_ps} \\
p_b &= a_b T^4,
\label{eq:bag_pb}
\end{align}
where $a_s$, $a_b$ and $V_s$ are positive constants with $a_s > a_b$.
s=symmetric, b=broken.


\section{More complex models}
The bag model includes severe approximations, and for realistic simulations a more complex model is needed.
The primary parameters for these models are
\begin{itemize}
    \item The phase transition parameters: percolation temperature, phase transition duration
    \item Bubble wall velocity
    \item Fraction of energy that is converted into fluid motion: $\kappa$
    \item Numerical prefactor from lattice simulations
\end{itemize}

Speed of sound may not be ultrarelativistic, especially in the broken phase.

\subsection{The $\nu$-model}
The $\nu$ model expands the bag model by allowing a change in the speed of sound of the broken phase.
\cite[eq. 38]{giese_2020}
\begin{align}
p_s &= \frac{1}{3} a_+ T^4 - \epsilon \\
p_b &= \frac{1}{3} a_- T^\nu \\
e_s &= a_+ T^4 + \epsilon \\
e_b &= \frac{1}{3} a_- (\nu - 1) T^\nu
\end{align}
The parameter $\nu$ can be written using the speed of sound of the broken phase as
\cite[eq. 39]{giese_2020}
\begin{equation}
\nu = 1 + \frac{1}{c_s^2}.
\end{equation}

\subsection{Model with cubic thermal effects}
Free energy
\todo{Where does this come from? How to extract the thermal effects?}
\cite[eq. 45]{giese_2020}
\begin{align}
\begin{split}
\textsc{F}(\phi, T) =
&- \frac{a_+}{3} T^4 \\
&+ \lambda \left( \phi^4 - 2E\phi^3 T + \phi^2 \left( E^2 T_\text{cr}^2 + c \left( T^2 - T_\text{cr}^2 \right) \right) \right) \\
&+ \frac{\lambda}{4} \left( c - E^2 \right)^2 T_\text{cr}^4
\end{split}
\end{align}
1
The minimum of the field potential is given by
\cite[eq. 46]{giese_2020}
\begin{equation}
\phi_\text{min} = \frac{3}{4} ET + \sqrt{\frac{T^2}{2}(\frac{9}{8}E^2 - c) - \frac{T_\text{cr}^2}{2} (E^2 - c)}
\end{equation}

\subsection{Model with a two-step phase transition}
Two scalar fields and two symmetry breakings. Examples: electroweak symmetry, $\mathbb{Z}_2$ symmetry.
Cubic term is neglected.
Pressures are of the form
\cite[eq. 47-48]{giese_2020}
\begin{align}
p_s(T) = \frac{1}{3}a_+ T^4 + (b_+ - c_+ T^2)^2 - b_-^2 \\
p_b(T) = \frac{1}{3}a_+ T^4 + (b_- - c_-T^2)^2 - b_-^2
\end{align}
Compare these to the equations \ref{eq:bag_ps}, \ref{eq:bag_pb}.



\section{Types of solutions}
\missingfigure{Figure of the three different types of relativistic combustion. \cite[fig. 14]{lecture_notes}}

Subsonic deflagrations

Detonations

Supersonic deflagrations (hybrids)


\section{Energy redistribution}
Efficiency factor $\kappa$, which is the fraction of energy that is converted into fluid motion.
\begin{equation}
\kappa = \frac{3}{\chi_w^3 \Delta \theta} \int d\chi \chi^2 w \gamma^2 v^2
\end{equation}
From solving the hydrodynamic equations of state of a single expanding bubble.

Kinetic energy fraction
\cites[eq. 7.36]{lecture_notes}[eq. 5]{giese_2020}
\begin{align}
K
&= \frac{\rho_\text{fl}}{\bar{e}}
&= \frac{1}{\mathcal{v} \bar{e}} \int d^3 x w \gamma^2 v^2,
\end{align}
where $\mathcal{v}$ is the volume we average over, and $\bar{e}$ is the mean energy density.
It should be noted that $0 < K < 1$.
\cite{giese_2020}

For a single bubble we have, using its volume and the energy density of the symmetric phase that the bubble replaces, the kinetic energy fraction
\cites[eq. 7.37]{lecture_notes}[eq. 5]{giese_2020}
\begin{equation}
K_1 = \frac{3}{\chi_w^3 e_s} \int d\chi \chi^2 w \gamma^2 v^2.
\end{equation}

The kinetic energy fraction can be determined from
\begin{itemize}
\item wall velocity $v_w$
\item the speeds of sound
\item strength parameter of the phase transition
\end{itemize}

Let us define the pseudotrace as
\cites[eq. 34]{giese_2020}[eq. 1]{giese_2021}
\begin{equation}
\overline{\theta} \equiv e - \frac{p}{c_{s,b}^2}.
\end{equation}

For detonations, the optimal phase transition strength is
\cites[eq. 34]{giese_2020}[eq. 1]{giese_2021}
\todo{Explain how this is defined}
\begin{equation}
\alpha_{\overline{\theta}} \equiv \frac{D \overline{\theta}(T_+)}{3w_+},
\end{equation}


\section{Extensions of the standard model}
Extensions of the Standard model \cite{caprini_detecting_2020}
(mentioned in \cite[p. 14]{lecture_notes})

\section{Noise background}
Non-thermal phenomena: GW production by preheating, when the inflaton decays to SM particles.
\cite{lecture_notes}
