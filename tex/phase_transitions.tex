\section{Todo notes}

Equations of state
\begin{itemize}
    \item Bag model \cite[eq. 7.33]{lecture_notes}
    \item Assumptions on speed of sound and constant parameters
    \item Giese's paper \cite{giese_2020}
    \item Enqvist, old paper \cite{enqvist_nucleation_1992}
\end{itemize}

Employment?
- "things that need to be done"
-> suitable for public dissemination
- write some short description,
    - ArXiv paper

2 papers by Giese, second one is more complete
Paper 1 \cite{giese_2020}
Paper 2 \cite{giese_2021}

TODO: understand these
\begin{itemize}
    \item Relativistic hydrodynamics from the review article (aka. lecture notes) \cite{lecture_notes}
    \item Reviews from other people (not just Hindmarsh)
    \item Article by Mazumdar \& White, especially section 7: relativistic combustion \cite{mazumdar_review_2019}
    \item The sound shell model paper
    \item Really understand what's going on! Compute with pen \& paper etc.
    \item Relativistic hydrodynamics
    \begin{itemize}
        \item The book by Luciano Rezzolla \& Olindo Zanotti \cite{rezzolla_relativistic_2013}
        \item Read the part I (some 300 pages at first)
        \item Contains more than necessary for the thesis
    \end{itemize}
    \item The phase transition section should have plenty of content before Christmas
    \item Enqvist's paper
    \begin{itemize}
        \item Good background: connects the equations of state to particle physics
        \item -> What kinds of equations of state would a particle theory produce
        \item Assumes ultrarelativisticity, which is no longer necessary due to the improvement of computational tools
    \end{itemize}
    \item Ultimately the equation of state can be derived from the "Thermal effective potential".
    \item Chemical potential is zero, as there is no conserved particle number.
    \item J-function, fB, lecture eq. 2.19 etc.
    \begin{itemize}
        \item These will be our first equation of state.
    \end{itemize}
\end{itemize}

"My job is to give the GWs the right energy-momentum tensor"

\section{Basics}
Electroweak symmetry breaking at $10^{-11}$ s, $100$ -- $1000$ GeV.
First-order phase transition
-> bubbles just below the critical temperature
-> gravitational waves

Standard model has a crossover, but many extensions first-order.

Matter-antimatter asymmetry -> net baryon number
(But isn't B an accidental symmetry, whereas B-L is a fundamental one?
How does this correspond to the neutrino content of the universe?)
\cite{lecture_notes}

The gravitational wave spectrum can be calculated from a few thermodynamic properties of ultrarelativistic matter, which can be computed from quantum field theory.
These parameters can be measured by LISA.



\section{Relativistic hydrodynamics}
% Statistical mechanics \cite{huang_statistical_1987}
% \cite{schroeder_thermal_2000}
% \cite[ch. 4]{lecture_notes}
% According to Hindmarsh it's OK to have references on the level of individual chapters and equations.

Our system of interest is an ultrarelativistic plasma.
This means, that the energy of the particles is much higher than their rest mass.
Therefore there is sufficient energy for new particles to be created,
and similarly existing particles will annihilate all the time.
Treating this kind of a system as a classical fluid is not sufficient,
and we need the mathematical machinery of general relativity.

In general relativity the matter and energy content of space are described by the \textbf{energy-momentum tensor} $T^{\mu \nu}$, also known as the stress-energy tensor.
In Minkowski space in Cartesian coordinates it's given as
\cites[eq. 4.17]{rasanen_gr_2022}[fig. 3.3]{rezzolla_relativistic_2013}
\begin{equation}
T_{\mu \nu} =
\begin{bmatrix}
e & -q_1 & -q_2 & -q_3 \\
-q_1 & p + \Pi_{11} & \Pi_{12} & \Pi_{13} \\
-q_2 & \Pi_{12} & p + \Pi_{22} & \Pi_{23} \\
-q_3 & \Pi_{13} & \Pi_{23} & p + \Pi_{33}
\end{bmatrix},
\end{equation}
where $q$ is the energy flux or momentum density and $\Pi_{\mu \nu}$ is known as the anisotropic stress, anisotropic pressure or momentum flux.
\href{https://physics.stackexchange.com/a/412067/298623}{In our case we assume the energy-matter content to be an ideal fluid in thermal equilibrium.}
There is no energy transfer to or from the fluid, and therefore
$\forall j=1,2,3: \quad T^{0j} = 0$.
As the fluid has no viscosity, it does not experience any shear stress, and therefore
$\forall i,j=1,2,3, \quad i \neq j: \quad T^{ij} = 0$.
This does not depend on the reference frame, so the tensor is diagonal in all reference frames.
Therefore $T^{ij} = p \delta^{ij}$.
These are satisfied only by the tensor $T^{00}=e, T^{jj}=p, \forall i \neq j: T^{ij}=0$.
It should be noted that the
\href{https://en.wikipedia.org/wiki/Einstein_notation}{Einstein notation} will be used for the indices throughout the thesis.
Greek indices are used for four-vectors and latin indices for three-vectors.
% The Einstein notation is absolute basics for theoretical physicists, but unfortunately the study programme of many of my applied/technical physicists friends doesn't involve it, so I'm mentioning it explicitly here so that they can google it.
This
\href{https://en.wikipedia.org/wiki/Stress\%E2\%80\%93energy\_tensor\#Stress\%E2\%80\%93energy\_of\_a\_fluid\_in\_equilibrium}{\textbf{energy-momentum tensor of an ideal fluid}}
can be broken in two components as
\cites[eq. 5.11, 5.23]{lecture_notes}[eq. 4.12]{rasanen_gr_2022}
% [eq. 3]{giese_2020}[eq. 4]{giese_2021}
\begin{align}
T^{\mu \nu}_f
&= (e+p) u^\mu u^\nu + p g^{\mu \nu}
\label{eq:ep_tensor} \\
&= w u^\mu u^\nu + p g^{\mu \nu}.
\end{align}

We are assuming the background space-time to be constant, and therefore the total \textbf{energy-momentum is conserved}.
In the language of general relativity this is
\begin{equation}
\nabla_\mu T^{\mu\nu} = 0.
\label{eq:ep_conservation}
\end{equation}
Here $\nabla_\mu$ is the covariant derivative, which in Minkowki space in Cartesian coordinates reduces to the partial derivative $\partial_\mu$.
To expand this equation in a way consistent with \cite[ch. 3.3]{rezzolla_relativistic_2013} we need to define the projection tensor
\cite[eq. 3.9]{rezzolla_relativistic_2013}
\begin{equation}
h_{\mu\nu} \equiv g_{\mu\nu} + u_\mu u_\nu,
\label{eq:projection_tensor}
\end{equation}
and the expansion scalar
\cite[eq. 3.13]{rezzolla_relativistic_2013}
\begin{equation}
\Theta \equiv h^{\mu\nu} \nabla_\nu u_\mu = \nabla_\mu u^\mu.
\end{equation}
Inserting these to \eqref{eq:ep_conservation} results in
% The specific enthalpy is not a meaningful quantity for an ultrarelativistic fluid, and is therefore not used here.
\begin{equation}
\nabla_\mu T^{\mu \lambda} = u^\lambda u^\mu \nabla_\mu (e+p) + (e+p) u^\mu \nabla_\mu u^\lambda + (e+p) \Theta u^\lambda + g^{\lambda\mu} \nabla_\mu p.
\end{equation}
\todo{Should enthalpy be introduced before this? It would make the e+p terms simpler.}
Relativistic acceleration is defined as
\begin{equation}
a_\nu = u^\mu \nabla_\mu u_\nu.
\end{equation}
It is orthogonal to the velocity, and therefore
\begin{equation}
a^\mu u_\mu = 0.
\end{equation}
Projecting using \eqref{eq:ep_conservation} and using these we get
\cite[eq. 3.54]{rezzolla_relativistic_2013}
\begin{equation}
\bm{h} \cdot \bm{\nabla} \cdot \bm{T}
= h^\nu_\lambda \nabla_\mu T^{\mu \lambda}
= (e+p) u^\mu \nabla_\mu u^\nu + h^\nu_\lambda g^{\lambda\mu} \nabla_\mu p
= 0.
\end{equation}
It should be noted that \cite{rezzolla_relativistic_2013} uses the specific enthalpy $h$ in their form these equations, but it is not a meaningful quantity for an ultrarelativistic plasma and is therefore not used here.
Dividing by $e+p$ we get the \textbf{relativistic Euler equations}
\cite[eq. 3.55]{rezzolla_relativistic_2013}
\begin{equation}
u^\mu \nabla_\mu u_\nu + \frac{1}{e+p} h^\mu_\nu \nabla_\mu p = 0.
\end{equation}
Similarly we can project along the direction of velocity,
\begin{equation}
\bm{u} \cdot \bm{\nabla} \cdot \bm{T} = u_\lambda \nabla_\mu T^{\mu\lambda} = 0.
\end{equation}
This results in the \textbf{energy-conservation equation}
\cite[eq. 3.57]{rezzolla_relativistic_2013}
\begin{equation}
u^\mu \nabla_\mu e + (e+p) \Theta = 0.
\end{equation}

For a one-dimensional flow in Cartesian coordinates the energy-momentum conservation of \eqref{eq:ep_conservation} can also be rewritten as
\begin{align}
\partial_t \left[ (e+pv^2) \gamma^2 \right] + \partial_x \left[ (e+p) \gamma^2 v \right] &= 0,
\label{eq:ep_conservation_1d_1} \\
\partial_t \left[ (e+p) \gamma^2 v \right] + \partial_x \left[ (ev^2 + p) \gamma^2 \right] &= 0
\label{eq:ep_conservation_1d_2}
\end{align}
by simply inserting \eqref{eq:ep_tensor} and using the normalisation of four-velocity $u_\mu u^\mu = -1$.
Let's then analyse a perturbation on a fluid that is at rest with $e_0, p_0$ and $v_0 = 0$.
To first order
\begin{equation}
e = e_0 + \delta e, \quad p=p_0 + \delta p, \quad v = v_0 + \delta v = \delta v.
\end{equation}
Substituting these into \eqref{eq:ep_conservation_1d_1} and \eqref{eq:ep_conservation_1d_2} and approximating to first order in the perturbation results in
\begin{align}
\partial_t (\delta e) + (e_0 + p_0) \partial_x (\delta v) = 0,
\label{eq:ep_conservation_rewritten_1} \\
\partial_t (\delta v) + \frac{1}{e_0 + p_0} \partial_x (\delta p) = 0.
\label{eq:ep_conservation_rewritten_2}
\end{align}
Taking the time derivative of \eqref{eq:ep_conservation_rewritten_1} and a space derivative of \eqref{eq:ep_conservation_rewritten_2}, we can combine the equations to the form
\begin{equation}
\partial_t^2 (\delta e) - \frac{\delta p}{\delta e} \partial_x^2(\delta e) = 0.
\end{equation}
This is the familiar wave equation, so the motion of the fluid has wave solutions that we call sound.
\cites[ch. 4.3]{rezzolla_relativistic_2013}[ch. 7.4]{lecture_notes}
The speed of sound is then defined as
\cites[eq. 2.168]{rezzolla_relativistic_2013}[eq. 13]{giese_2020}[eq. 3]{giese_2021}
\begin{align}
c_s^2
&\equiv \frac{dp}{de} \\
&= \frac{dp/dT}{de/dT}
\end{align}
At the ultrarelativistic limit $e=3p$ and therefore $c_s^2=\frac{1}{3}$.
\todo{Write here why}

\iffalse
The
\href{https://en.wikipedia.org/wiki/Clausius\%E2\%80\%93Clapeyron_relation}{Clausius-Clapeyron relation} states that
% According to Hindmarsh it's OK to have hyperlinks to unofficial sources such as Wikipedia.
\cite[eq. 5.47, 5.48]{schroeder_thermal_2000}
\begin{equation}
\frac{dp}{dT} = \frac{L}{T \Delta V} = \frac{\Delta S}{\Delta V}.
\end{equation}
\fi

Proper discussion of relativistic fluids also requires various thermodynamic quantities.
\todo{Derive entropy density properly}
% Let us start from the thermodynamic identity of classical thermal physics
Perturbing around a constant $U$ (effectively setting $dU=0$) gives the entropy density
\todo{Is this the correct explanation?}
\cite[p. 23]{lecture_notes}
\begin{equation}
s \equiv \frac{dS}{dV} = \frac{dp}{dT}.
\label{eq:entropy_density}
\end{equation}

Useful formulas for other thermodynamic quantities can be obtained from the thermodynamic identity of classical thermal physics.
\cites[eq. 2.136]{rezzolla_relativistic_2013}[eq. 3.68]{schroeder_thermal_2000}
\begin{equation}
dU = TdS - pdV + \mu dN,
\label{eq:thermodynamic_identity}
\end{equation}
where $U$ is internal energy, $T$ is temperature, $p$ is pressure, $V$ is volume, $\mu$ is chemical potential and $N$ is the number of particles.
The fluids in our case are ultrarelativistic, and therefore effectively $T >> \mu$,
and we can neglect $dN$.

Inserting \eqref{eq:entropy_density} to \eqref{eq:thermodynamic_identity} gives the \textbf{energy density}
\begin{equation}
e \equiv \frac{dU}{dV} = T \frac{\partial p}{\partial T} - p.
\end{equation}
This in turn can be inserted to the definition of enthalpy
\cite[eq. 1.51]{schroeder_thermal_2000}
\begin{equation}
H \equiv E + pV
\end{equation}
to get the \textbf{enthalpy density}
\begin{align}
w
&\equiv \frac{dH}{dV} \\
&= e+p \\
&= T \frac{\partial p}{\partial T} \\
&= Ts.
\end{align}
\iffalse
Additionally we need to define the \textbf{specific enthalpy} \cite[eq. 2.141]{rezzolla_relativistic_2013}
\begin{equation}
h = \frac{e+p}{\rho},
\label{eq:specific_enthalpy}
\end{equation}
where $\rho$ is the rest-mass density, and the \textbf{entropy current}
% Rest-mass density is defined in eq. 2.123
\fi
Additionally we need to define the \textbf{entropy current}
\cite[p. 23]{lecture_notes}
\begin{equation}
S^\mu \equiv su^\mu.
\end{equation}



\section{Field-fluid system}
Based on \cite{moore_pt_1995}.

In the electroweak phase transition the fluid is coupled to a scalar field with the energy-momentum tensor
\cites[eq. 2.9]{hindmarsh_gw_pt_2019}
\begin{equation}
T_\phi^{\mu \nu}
= \partial^\mu \partial^\nu \phi
- g^{\mu \nu} \left(\frac{1}{2} (\partial \phi)^2 + V_0 (\phi) \right)
\end{equation}

The total energy-momentum is conserved,
\cite[eq. 5.17]{lecture_notes}
\begin{equation}
\partial_\mu T^{\mu \nu} = \partial_\mu (T_f^{\mu \nu} + T_\phi^{\mu \nu}) = 0.
\end{equation}



\section{Relativistic combustion}
This chapter of the lecture notes is largely based on the references
% According to Hindmarsh, abbreviating "references" as "refs" is OK in the thesis.
\cites{hindmarsh_gw_pt_2019}{espinosa_energy_2010}.
\todo{Add more content from \cite[ch. 6]{mazumdar_review_2019}.}

If the wall velocity is constant, \eqref{eq:ep_conservation} also holds in the wall frame.
Therefore for a wall moving in the z-direction
\cite[eq. 7]{espinosa_energy_2010}
\begin{equation}
\partial_z T^{zz} = \partial_z T^{z0} = 0.
\end{equation}
Inserting the definition of $T$ from \eqref{eq:ep_tensor} results in the bubble wall junction conditions
\cites[eq. 7.22]{lecture_notes}[eq. B.2-3]{hindmarsh_gw_pt_2019}
% This is almost trivial.
\begin{align}
w_- \tilde{\gamma}_-^2 \tilde{v}_- &= w_+ \tilde{\gamma}_+^2 \tilde{v}_+, \\
w_- \tilde{\gamma}_-^2 \tilde{v}_-^2 + p_- &= w_+ \tilde{\gamma}_+^2 \tilde{v}_+^2 + p_+
\end{align}
These can be rearranged as
\cites[eq. 7.32]{lecture_notes}[eq. 6-7]{giese_2020}
\todo{How to prove this step? My attempts of manual proof have failed.}
\begin{align}
\tilde{v}_+ \tilde{v}_- = \frac{p_+ - p_-}{e_+ - e_-} \\
\frac{\tilde{v}_+}{\tilde{v}_-} = \frac{e_- + p_+}{e_+ + p_-}
\label{eq:junction_ep}
\end{align}
Further simplification of these requires additional definitions
that we will now introduce.

The trace of the energy-momentum tensor of \eqref{eq:ep_tensor} is
\begin{align}
\text{tr} (T^{\mu \nu})
&= T^\mu_\mu \\
&= g_{\mu \nu} T^{\mu \nu} \\
&=3p - e
\end{align}
We can quantify its difference from zero with the trace anomaly
\cites[eq. 7.24]{lecture_notes}[eq. 28]{giese_2020}
\begin{equation}
\theta = \frac{1}{4}(e-3p).
\end{equation}
The definition of the trace anomaly varies by a factor of $\frac{1}{4}$ depending on the source as a matter of convention.
The difference of the trace anomalies just ahead and behind of the wall is denoted as
\begin{align}
\Delta \theta
&\equiv \theta_+(w_+) - \theta_-(w_-) % \\
% &= \theta_s(w_s) - \theta_b(w_b)
\end{align}
With this we can define the transition strength
\begin{equation}
\alpha_+ = \frac{4 \Delta \theta}{3 w_+}.
\end{equation}
The transition strength at nucleation temperature is defined slightly differently as
\cite[eq. 2.11]{hindmarsh_gw_pt_2019}
\begin{equation}
\alpha_n = \frac{4}{3} \frac{\theta_+(w_n) - \theta_-(w_n)}{w_n}.
\end{equation}
We can also define the enthalpy ratio
\begin{equation}
r = \frac{w_+}{w_-}.
\end{equation}
With these definitions the junction conditions of \eqref{eq:junction_ep} can be simplified as
\begin{equation}
\tilde{v}_+ \tilde{v}_- = \frac{1-(1-3\alpha_+)r}{3-3(1+\alpha_+)r},
\quad
\frac{\tilde{v}_+}{\tilde{v}_-} = \frac{3+(1-3\alpha_+)r}{1+3(1+\alpha_+)r}.
\end{equation}
% If a superlative is without a noun, there is no "the".
This is easiest to prove in the reverse direction.
To further get expressions for $\tilde{v}_+$ and $\tilde{v}_-$ that are independent of $r$, we can extract $r$ from both equations and mark those as equal, resulting in the equation
\begin{equation}
\left( \frac{1}{\tilde{v}_-} + 3 \tilde{v}_- \right) \tilde{v}_+ - 3(1+\alpha_+)\tilde{v}_+^2 = \alpha_+ - 1.
\end{equation}
% Derivation: Mathematica Eliminate[] and a bit of simplification.
This is of second order in both $\tilde{v}_+$ and $\tilde{v}_-$ and can be solved as \cite[eq. B.6, B.7]{hindmarsh_gw_pt_2019}
\begin{align}
\tilde{v}_+ &= \frac{1}{2(1+\alpha_+)}\left[ \left(\frac{1}{3\tilde{v}_-}+\tilde{v}_-\right) \pm \sqrt{\left(\frac{1}{3\tilde{v}_-} - \tilde{v}_- \right)^2 + 4\alpha_+^2 + \frac{8}{3} \alpha_+} \right]
\label{eq:v_tilde_plus}
\\
\tilde{v}_- &= \frac{1}{2} \left[ \left( (1+\alpha_+)\tilde{v}_+ + \frac{1-3\alpha_+}{3\tilde{v}_+} \right) \pm \sqrt{\left((1+\alpha_+)\tilde{v}_+ + \frac{1-3\alpha_+}{3\tilde{v}_+} \right)^2 - \frac{4}{3}} \right)
\label{eq:v_tilde_minus}
\end{align}
Both of these are basic second-order solutions, but in \eqref{eq:v_tilde_plus} the discriminant has been simplified.


We can obtain useful continuity equations from the conservation of energy-momentum.
We start by defining two vectors, the fluid 4-velocity
\begin{equation}
u^\mu = \gamma(-1, \overrightarrow{v})^\mu
\end{equation}
and its orthonormal vector
\begin{equation}
\bar{u}^\mu = \gamma(-v, \frac{\overrightarrow{v}}{v})^\mu.
\end{equation}
These form an orthonormal basis.
It should be noted, that for orthonormal vectors
\begin{align}
% Note that the lecture notes use ubar here.
u_\mu u^\mu &= -1, \\
u_\mu \bar{u}^\mu &= 0.
\end{align}
By projecting the conservation equation to this basis we get
\cite[eq. 7.28-7.29]{lecture_notes}
\begin{align}
0 = u_\mu \partial_\nu T^{\mu \nu} = -\partial_\mu (w u^\mu) + u^\mu \partial_\mu p \\
0 = \bar{u} \partial_\nu T^{\mu \nu} = w \bar{u}^\nu u^\mu \partial_\mu u_\nu + \bar{u}^\mu \partial_\mu p
\end{align}

Dimensionless coordinate
\begin{equation}
\xi = \frac{r}{t}
\end{equation}

Hydrodynamic equations, aka. continuity equations
\todo{Todo: Derive these using Espinoza's article, and see the handwritten notes by Hidmarsh. It's longer than it looks.}
\cites[eq. 7.30-7.31]{lecture_notes}[eq. 5]{giese_2021}
\begin{align}
\frac{dv}{d\xi} &= \frac{2v(1-v^2)}{\xi(1-v\xi)} \left( \frac{\mu(\xi,v)^2}{c_s^2} - 1 \right)^{-1} \\
\frac{dw}{d\xi} &= w \left( 1 + \frac{1}{c_s^2} \right) \gamma^2 \mu(\xi,v) \frac{dv}{d\xi}
\end{align}

It should be highlighted that everything in this chapter is independent of the equation of state introduced in the next chapter.



\section{The bag model}
The equation of state is the function $p(T)$ or $p(w)$ that characterises the behaviour of the fluid as a function of temperature $T$ or enthalpy $w$.
One of the simplest equations of state is the bag model, where
\cites[eq. 7.33]{lecture_notes}[eq. 8-9]{giese_2020}
\begin{align}
p_s &= a_s T^4 - V_s
\label{eq:bag_ps} \\
p_b &= a_b T^4,
\label{eq:bag_pb}
\end{align}
where $a_s$, $a_b$ and $V_s$ are positive constants with $a_s > a_b$.
s=symmetric, b=broken.

In this model the speed of sound simplifies as
\begin{equation}
c_s^2 = \frac{dp}{de} = 1/3,
\end{equation}
This corresponds to assuming the bth phases to be ultrarelativistic.

The name of the bag model originates from quantum chromodynamics (QCD), where it's used to describe the proton as a bag of quarks.
\todo{check this}


\section{More complex models}
The bag model includes severe approximations, and for realistic simulations a more complex model is needed.
The primary parameters for these models are
\begin{itemize}
    \item The phase transition parameters: percolation temperature, phase transition duration
    \item Bubble wall velocity
    \item Fraction of energy that is converted into fluid motion: $\kappa$
    \item Numerical prefactor from lattice simulations
\end{itemize}

Speed of sound may not be ultrarelativistic, especially in the broken phase.

\subsection{The constant sound speed model}
The constant sound speed model, also known as the $\mu, \nu$ model, expands the bag model by allowing a change in the speed of sound of the broken phase.
\cites[eq. 15]{giese_2021}[eq. 38]{giese_2020}
\begin{align}
p_s &= \frac{1}{3} a_+ T^\mu - \epsilon \\
p_b &= \frac{1}{3} a_- T^\nu \\
e_s &= a_+ T^\mu + \epsilon \\
e_b &= \frac{1}{3} a_- (\nu - 1) T^\nu
\end{align}
The parameters $a_+$ and $a_-$ are proportional to the number of relativistic degrees of freedom in the symmetric and broken phases, and $\epsilon$ is the temperature-independent vacuum energy that is released in the phase transition.
The parameters $\mu$ and $\nu$ are determined by the sound speeds in the symmetric and broken phases:
\cites[eq. 16]{giese_2021}[eq. 39]{giese_2020}
\begin{align}
\mu = 1 + \frac{1}{c_{s,s}^2}, \\
\nu = 1 + \frac{1}{c_{s,b}^2}.
\end{align}
\footnote{Please note that there is a typo in \cite[eq. 15]{giese_2021}. There the 4 should be a $\mu$.}

\subsection{Model with cubic thermal effects}
Free energy
\todo{Where does this come from? How to extract the thermal effects?}
\cite[eq. 45]{giese_2020}
\begin{align}
\begin{split}
\textsc{F}(\phi, T) =
&- \frac{a_+}{3} T^4 \\
&+ \lambda \left( \phi^4 - 2E\phi^3 T + \phi^2 \left( E^2 T_\text{cr}^2 + c \left( T^2 - T_\text{cr}^2 \right) \right) \right) \\
&+ \frac{\lambda}{4} \left( c - E^2 \right)^2 T_\text{cr}^4
\end{split}
\end{align}

The minimum of the field potential is given by
\cite[eq. 46]{giese_2020}
\begin{equation}
\phi_\text{min} = \frac{3}{4} ET + \sqrt{\frac{T^2}{2}(\frac{9}{8}E^2 - c) - \frac{T_\text{cr}^2}{2} (E^2 - c)}
\end{equation}

\subsection{Model with a two-step phase transition}
Two scalar fields and two symmetry breakings. Examples: electroweak symmetry, $\mathbb{Z}_2$ symmetry.
Cubic term is neglected.
Pressures are of the form
\cite[eq. 47-48]{giese_2020}
\begin{align}
p_s(T) = \frac{1}{3}a_+ T^4 + (b_+ - c_+ T^2)^2 - b_-^2 \\
p_b(T) = \frac{1}{3}a_+ T^4 + (b_- - c_-T^2)^2 - b_-^2
\end{align}
Compare these to the \eqref{eq:bag_ps}, \eqref{eq:bag_pb}.



\section{Types of solutions}
\missingfigure{Figure of the three different types of relativistic combustion. \cite[fig. 14]{lecture_notes}}

Subsonic deflagrations

Detonations

Supersonic deflagrations (hybrids)


\section{Energy redistribution}
Efficiency factor $\kappa$, which is the fraction of energy that is converted into fluid motion.
\begin{equation}
\kappa = \frac{3}{\xi_w^3 \Delta \theta} \int d\xi \xi^2 w \gamma^2 v^2
\end{equation}
From solving the hydrodynamic equations of state of a single expanding bubble.

Kinetic energy fraction
\cites[eq. 7.36]{lecture_notes}[eq. 5]{giese_2020}
\begin{align}
K
&= \frac{\rho_\text{fl}}{\bar{e}}
&= \frac{1}{\mathcal{v} \bar{e}} \int d^3 x w \gamma^2 v^2,
\end{align}
where $\mathcal{v}$ is the volume we average over, and $\bar{e}$ is the mean energy density.
It should be noted that $0 < K < 1$.
\cite{giese_2020}

For a single bubble we have, using its volume and the energy density of the symmetric phase that the bubble replaces, the kinetic energy fraction
\cites[eq. 7.37]{lecture_notes}[eq. 5]{giese_2020}
\begin{equation}
K_1 = \frac{3}{\xi_w^3 e_s} \int d\xi \xi^2 w \gamma^2 v^2.
\end{equation}

The kinetic energy fraction can be determined from
\begin{itemize}
\item wall velocity $v_w$
\item the speeds of sound
\item strength parameter of the phase transition
\end{itemize}

Let us define the pseudotrace as
\cites[eq. 34]{giese_2020}[eq. 1]{giese_2021}
\begin{equation}
\overline{\theta} \equiv e - \frac{p}{c_{s,b}^2}.
\end{equation}

For detonations, the optimal phase transition strength is
\cites[eq. 34]{giese_2020}[eq. 1]{giese_2021}
\todo{Todo: Explain how this is defined}
\begin{equation}
\alpha_{\overline{\theta}} \equiv \frac{D \overline{\theta}(T_+)}{3w_+},
\end{equation}


\section{Extensions of the standard model}
Extensions of the Standard model \cite{caprini_detecting_2020}
(mentioned in \cite[p. 14]{lecture_notes})

\section{Noise background}
Non-thermal phenomena: GW production by preheating, when the inflaton decays to SM particles.
\cite{lecture_notes}
