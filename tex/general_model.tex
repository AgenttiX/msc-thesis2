\section{General equation of state}
\label{general_eq}
The equation of state is the function $p(T)$ or $p(w)$ that characterises the behaviour of the fluid as a function of temperature $T$ or enthalpy $w$.
It should be highlighted that everything up to this point has been independent of the choice of the equation of state.
We will now first describe the derivation for the general equation of state of an ultrarelativistic plasma,
and then introduce approximate models.

\iffalse
In general the pressure of the ultrarelativistic ASDF
\begin{equation}
p(T,\phi) = - \sum_B f_B (m(\phi),T) - \sum_F f_F (m(\phi),T)
\end{equation}
\fi

The energy density at a point is the sum of the energies of all particles at that point.
Therefore, the energy density in the position space can be obtained by integrating over the momentum space as
\cite[eq. 4.10]{lecture_notes}
\begin{equation}
e(x) = \int \frac{d^3 p}{(2 \pi)^3} E(p) f(\vec{p}, x).
\end{equation}
% https://physics.stackexchange.com/a/141737/
% The factor $(2\pi)^3$ normalises the momentum space.
Similarly we can obtain the other components of the energy-momentum tensor in the position space as
\cite[eq. 4.13]{lecture_notes}
\begin{equation}
T^{\mu \nu}(x) = \int \frac{d^3 p}{(2 \pi)^3} \frac{p^\mu p^\nu}{p^0} f(\vec{p},x).
\end{equation}
To give an intuitive explanation for this we can note that $p^0 = E = \gamma m_0$ and $p^i = \gamma m_0 v^i$.
Therefore, $\frac{p^i}{p^0} = v^i$,
and the factor $\frac{p^i p^j}{p^0} = p^i v^j$ denotes the amount of momentum being transported by the moving plasma.
By inserting this to the definition of pressure~\eqref{eq:pressure_from_ep_tensor} we get
\begin{equation}
p(x) = \frac{1}{3} \int \frac{d^3 p}{(2 \pi)^3} \frac{|\vec{p}|^2}{E} f(\vec{p},x)
\end{equation}
This notation is rather confusing, as the $p$ on the left refers to the pressure,
and all $p$ on the right refer to the momentum.

The particle distribution functions for fermions and bosons are
\cite[eq. 4.6]{lecture_notes}
\begin{equation}
f(\vec{p}) = \frac{1}{e^\frac{E-\mu}{T} \pm 1},
\end{equation}
where $+$ gives the Fermi-Dirac distribution for fermions and $-$ gives the Bose-Einstein distribution for bosons.
Let us approximate that the fluid consists entirely of bosons,
and that they are ultrarelativistic: $E(\vec{p}) = \sqrt{\vec{p}^2 + m^2} \approx p$.
Particle number is not conserved in an ultrarelativistic plasma, and therefore $\mu = 0$.
Let us also remember that we can convert the integral to 1D using the area of a sphere $A(r) = 4\pi r^2$.
With these we get
\begin{align}
e &= \frac{1}{2 \pi^2} \int_0^\infty \frac{p^3 dp}{e^\frac{p}{T} - 1},
\label{eq:ultrarelativistic_e} \\
p &= \frac{1}{6 \pi^2} \int_0^\infty \frac{p^3 dp}{e^\frac{p}{T} - 1}.
\label{eq:ultrarelativistic_p}
\end{align}
These contain integrals of the form
\cite[eq. B.36]{schroeder_thermal_2000}
\begin{equation}
\int_0^\infty \frac{x^n}{e^x - 1} = \Gamma(n+1) \zeta(n+1),
\end{equation}
where $\Gamma$ is the
gamma function, which for integers is $\Gamma(n) = (n-1)!$, and $\zeta$ is the Riemann zeta function.
It can be shown that $\zeta(4) = \frac{\pi^4}{90}$.
\cite[prob. B.19]{schroeder_thermal_2000}
At this ultrarelativistic limit we can see from eq.~\eqref{eq:ultrarelativistic_e} and~\eqref{eq:ultrarelativistic_p} that $e = 3p$, and therefore using eq.~\eqref{eq:cs2_compact} that $c_s^2 = \frac{1}{3}$.

To account for the various particle species in the plasma and their non-ultrarelativistic behaviour,
we need to multiply with the degrees of freedom $g_e$ and $g_p$,
which are weighted sums over the contributions of different particle species.
To also account for the field, we need to add its potential $V$ to the final result
\cite[eq. S12]{borsanyi_lattice_2016}
\begin{align}
e(T,\phi) &= \frac{\pi^2}{30} g_e(T) T^4 + V(T,\phi),
\label{eq:e_general} \\
p(T,\phi) &= \frac{\pi^2}{90} g_p(T) T^4 - V(T,\phi).
\label{eq:p_general}
\end{align}
This means that we assume that all particles get their masses from the potential.

With eq.~\eqref{eq:enthalpy_sum},~\eqref{eq:enthalpy_entropy},~\eqref{eq:e_general} and~\eqref{eq:p_general} we can obtain the entropy density as
\cite[eq. S12]{borsanyi_lattice_2016}
\begin{equation}
s(T,\phi) = \frac{2\pi^2}{45} g_s(T) T^3,
\label{eq:s_general}
\end{equation}
where the degrees of freedom for the entropy density are
\begin{equation}
g_s = \frac{1}{4} (3g_e + g_p).
\end{equation}
Often only $g_e$ and $g_s$ are provided, and $g_p$ is obtained with
\begin{equation}
g_p = 4g_s - 3g_e.
\end{equation}
For a free scalar particle $g_e = g_p = g_s = 1$, which is known as the Stefan-Boltzmann limit.
\iffalse
due to its reminiscence to the Stefan-Boltzmann law $j^* = \sigma T^4$,
which relates the power radiated by a black body to its temperature with the Stefan-Boltzmann constant $\sigma$.
\fi
