\iffalse
Raise the interest of the reader!

Tip for style: discuss the same things from another point of view (in the light of the results) in the conclusion.
-> Binds the thesis together.

\begin{itemize}
    \item Background for the choice of theme
    \item A discussion of the research question or thesis statement
    \item A schematic outline of the thesis
\end{itemize}

Research question: how to simulate the gravitational spectra caused by different equations of state with PTtools?
\fi

The cosmic microwave background (CMB) has been our primary source of information on the earliest moments of the universe.
It is the light from the time the universe first became transparent to light, when the protons and electrons combined to form atoms.
However, this occurred when the universe was about 400 000 years old \todo{Or more accurate number?}, and therefore so far we haven't been able to make direct observations from earlier events.
This is about to change in the 2030s, when the Laser Interferometer Space Antenna (LISA) will be launched.
It is a space probe that consists of three satellites arranged in an equilateral triangle, where each side is connected by a 2.5 million kilometer long laser beam.
LISA is designed to detect gravitational waves,
for which the universe has been transparent ever since gravity separated from the other three fundamental interactions.
Therefore with gravitational waves we can make direct observations of events that occurred early in the Big Bang.
Possible sources of gravitational waves in the early universe include cosmological phase transitions, cosmic strings and inflation.
In this thesis we investigate the cosmological gravitational wave background generated by cosmological phase transitions.

In the Standard Model the Higgs transition is a cross-over instead of first-order,
and therefore there is no sharp phase boundary, and no generation of gravitational waves.
However, we know that the Standard Model is not a complete description of physics,
but has to be extended to account for various observations such as neutrino oscillations, matter-antimatter asymmetry and dark matter.
In many of these extensions of the Standard Model, the Higgs transition is a first-order phase transition,
and therefore results in a gravitational wave signal that we can measure.
This makes LISA a direct experiment that can distinguish between different extensions of the Standard Model.
The temperatures in the early universe were so extreme that they are extremely difficult to replicate in particle colliders,
and therefore LISA is capable of probing conditions beyond the scope of existing particle colliders.

First-order phase transitions proceed by the nucleation, expansion and collision of bubbles.
The situation is quite similar to that of a water kettle,
except that the temperature is decreasing.
As the universe cools down,
there occur locations where the plasma fluid spontaneously changes phase.
This phase transition releases energy, which pushes nearby areas to change to the new phase as well.
This creates a spherically expanding domain wall: a bubble.
As the bubbles expand, they eventually collide and merge.
However, this is not the end of the story, as the domain walls have caused the fluid to move with them,
and these waves persist beyond the end of the phase transition.
These waves are strong enough to cause the space-time itself to ripple with them,
and these are the gravitational waves that, if they have occurred, have persisted to the present day.
However, since they have occurred in the very early universe, they have redshifted to very long wavelengths.
This is why the existing LIGO gravitational wave insufficient,
as since it's a ground-based detector, its arms have been to short to detect these long-wavelength gravitational waves,
and the 2.5 million kilometer long lasers of the LISA are required.

The spectrum of the gravitational waves depends on several parameters of the phase transition.

The nucleation temperature $T_n$
The phase transition strength at the nucleation temperature $\alpha_n$
The wall speed $v_{\text{wall}}$ is determined by the friction of the fluid.
The transition rate parameter $\beta$
The speed of sound $c_s$
\todo{Explain where these come from?}

In the vast majority of the studies so far the speed of sound $c_s$ has been assumed to be that of ultrarelativistic plasma: $c_s = \frac{1}{\sqrt{3}}$.
However, in this thesis we investigate more realistic scenarios, where the plasma is not fully ultrarelativistic,
but instead it has degrees of freedom $g(T)$ that are dependent on the temperature and the potential of the field $V(T,\phi)$ depends on the temperature and the phase, and therefore the speed of sound $c_s(T,\phi)$ is dependent on the temperature and the phase.
This complicates the numerical simulation of the fluid profile of the bubbles significantly.
For these simulations the simulation framework PTtools has been extended to account for these more complex models.
PTtools is available on GitHub \cite{pttools}.
\todo{Publish PTtools}

The LaTeX and data analysis code of this thesis are available online \cite{thesis_source}.
This thesis is and these resources are licensed with
\href{Creative Commons Attribution 4.0 International}{https://creativecommons.org/licenses/by/4.0/}.

TODO: add sources
