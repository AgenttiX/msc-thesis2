\subsection{Model with cubic thermal effects}
Free energy
\todo{Where does this come from? How to extract the thermal effects?}
\cite[eq. 45]{giese_2020}
\begin{align}
\begin{split}
\textsc{F}(\phi, T) =
&- \frac{a_+}{3} T^4 \\
&+ \lambda \left( \phi^4 - 2E\phi^3 T + \phi^2 \left( E^2 T_\text{cr}^2 + c \left( T^2 - T_\text{cr}^2 \right) \right) \right) \\
&+ \frac{\lambda}{4} \left( c - E^2 \right)^2 T_\text{cr}^4
\end{split}
\end{align}

The minimum of the field potential is given by
\cite[eq. 46]{giese_2020}
\begin{equation}
\phi_\text{min} = \frac{3}{4} ET + \sqrt{\frac{T^2}{2}(\frac{9}{8}E^2 - c) - \frac{T_\text{cr}^2}{2} (E^2 - c)}
\end{equation}


\subsection{Model with a two-step phase transition}
Two scalar fields and two symmetry breakings. Examples: electroweak symmetry, $\mathbb{Z}_2$ symmetry.
Cubic term is neglected.
Pressures are of the form
\cite[eq. 47-48]{giese_2020}
\begin{align}
p_s(T) = \frac{1}{3}a_+ T^4 + (b_+ - c_+ T^2)^2 - b_-^2 \\
p_b(T) = \frac{1}{3}a_+ T^4 + (b_- - c_-T^2)^2 - b_-^2
\end{align}
Compare these to the \eqref{eq:bag_ps}, \eqref{eq:bag_pb}.


\subsection{Standard Model}
In the Standard Model the electroweak phase transition is not of first order but a crossover,
and therefore there is no bubble nucleation.
However, it serves as an important baseline.
The $g_e$ and $g_s$ data for the Standard Model is given by \cite{borsanyi_lattice_2016}.
