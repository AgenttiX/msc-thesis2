\iffalse
Motivation
\begin{itemize}
    \item Standard model: matches observations but lacks important explanations
    \item Baryon asymmetry
    \item GWs: a window to physics beyond the SM
    \item By phase transitions
    \item LISA project
\end{itemize}

The thesis
\begin{itemize}
    \item Highlights from the work, especially the title / thesis statement
    \item Findings
    \item Conclusion
\end{itemize}

This is among the last sections to be written.
\fi

The Standard Model of particle physics has proven to be remarkably accurate in various collider experiments.
However, it lacks explanations for some observed phenomena such as baryon asymmetry: Why is there more matter than antimatter?
The answer could be in cosmological phase transitions that have occurred early in the Big Bang.

Up to this date our earliest source of observational data has been the cosmic microwave background (CMB).
However, it was formed quite late in the Big Bang when the universe was about 400 000 years old.
This was the first time the universe became transparent to electromagnetic radiation,
and therefore the earliest time that we can probe with such experiments.
However, the universe has been transparent to gravitational waves ever since the gravitational force separated from the other three fundamental forces early in the Big Bang.
If the cosmological phase transitions that have occurred in the Big Bang have been of first order,
then they have caused gravitational waves that could be observable today.
To find out whether such phase transitions have occurred,
ASDFASDF

The Laser Interferometer Space Antenna (LISA) space probe will be launched in the 2030s to answer this question.
