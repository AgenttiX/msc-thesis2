The Standard Model of particle physics has proven to be remarkably accurate in various collider experiments,
but lacks explanations for some observed phenomena such as baryon asymmetry: Why is there more matter than antimatter?
The extensions of the Standard Model provide possible solutions to these questions,
but the energy scales required to distinguish between them are difficult to achieve in colliders.
However, many of these extensions also result in cosmological phase transitions that have occurred early in the Big Bang during the electroweak symmetry breaking at around $10^{-11} \text{s}$ or earlier.

If these phase transitions have been of first order,
they have created a stochastic background of gravitational waves that can be directly observed today.
The Laser Interferometer Space Antenna (LISA) space probe will be launched in the 2030s to study gravitational waves and
to see whether such a stochastic background exists.
If we see a signal of cosmological origin,
we need to be able to deduce the parameters of the phase transition from the gravitational wave signal to understand the physics behind it.
To accomplish this, we need simulations of the gravitational wave spectra with various parameters.
One such important parameter is the sound speed $c_s$.

The vast majority of existing simulations have been based on the bag model equation of state,
which assumes the ultrarelativistic sound speed $c_s =\frac{1}{\sqrt{3}}$ for both phases.
This was also the case for the PTtools phase transition simulation framework developed by Hindmarsh et al.
In this thesis PTtools has been extended to include support for arbitrary equations of state and therefore for a temperature- and phase-dependent sound speed $c_s(T,\phi)$.
Since the sound speed is such an integral part of the hydrodynamic equations,
this required a nearly complete rewrite and significant extension of the code.
The code has also been sped up considerably by the use of the Numba JIT compiler, various other optimisations and parallelisation,
and made conformant to modern coding standards.

PTtools was tested with the constant sound speed model, in which the sound speed is a constant for each phase.
The sound speed was shown to have a significant effect on the resulting gravitational wave spectrum,
especially when changing the sound speed resulted in a change in the type of the solution.
This has laid the groundwork for simulating cosmological phase transitions with realistic equations of state based on the extensions of the Standard Model.
This will result in gravitational wave spectra that can be used in the LISA data analysis pipeline to search for the existence and parameters of a first-order phase transition in the early universe.
