\iffalse
Motivation
\begin{itemize}
    \item Standard model: matches observations but lacks important explanations
    \item Baryon asymmetry
    \item GWs: a window to physics beyond the SM
    \item By phase transitions
    \item LISA project
\end{itemize}

The thesis
\begin{itemize}
    \item Highlights from the work, especially the title / thesis statement
    \item Findings
    \item Conclusion
\end{itemize}

This is among the last sections to be written.
\fi

The Standard Model of particle physics has proven to be remarkably accurate in various collider experiments.
However, it lacks explanations for some observed phenomena such as baryon asymmetry: Why is there more matter than antimatter?
The answer could be in cosmological phase transitions that have occurred early in the Big Bang.

Up to this date our earliest source of observational data has been the cosmic microwave background (CMB).
However, it was formed quite late in the Big Bang when the universe was about 400 000 years old.
This was the first time the universe became transparent to electromagnetic radiation,
and therefore the earliest time that we can probe with such experiments.
However, the universe has been transparent to gravitational waves ever since the gravitational force separated from the other three fundamental forces early in the Big Bang.
If the cosmological phase transitions that have occurred in the Big Bang have been of first order,
then they have caused gravitational waves that could be observable today.
To find out whether such phase transitions have occurred,
ASDFASDF

The Laser Interferometer Space Antenna (LISA) space probe will be launched in the 2030s to answer this question.

The sound speed is an important parameter for ASDF

The equation of state which determines the pressure $p(T,\phi)$ as a function of the temperature and phase is an important parameter.

However, the vast majority of existing simulations have been based on the bag model equation of state, which assumes the ultrarelativistic sound speed $c_s =\frac{1}{\sqrt{3}}$ for both phases.
This was also the case for the PTtools phase transition simulation framework.
In this thesis PTtools was extended \todo{Should I use "I" or "we" instead of passive?} to include support for arbitrary equations of state and therefore for a temperature- and phase-dependent speed of sound $c_s(T,\phi)$.
Since the sound speed is such an integral part of the hydrodynamic equations,
this required a nearly complete rewrite and significant extension of the code.

PTtools was tested with the constant sound speed model, in which the sound speed is a constant for each phase.
The sound speed was shown to have a significant effect on the resulting gravitational wave spectrum,
especially when changing the sound speed resulted in a change in the type of the solution.
This has laid the groundwork for simulating cosmological phase transitions with realistic equations of state based on the extensions of the Standard Model.
This will result in gravitational wave spectra that can be used in the LISA data analysis pipeline to search for the existence and parameters of a first-order phase transition in the early universe.

TODO Something about the detectability and signal-to-noise ratio?
