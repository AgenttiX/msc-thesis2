In this thesis the gravitational wave spectrum from the first order phase transition is calculated using the Sound Shell Model.
We shall start by rigorously deriving the necessary equations based on the section 3 of \cite{hindmarsh_gw_pt_2019}.

First of all we assume that the time scale of the phase transition is much shorter than the Hubble time.
Therefore we can neglect the expansion of the universe and assume that the background is Minkowski spacetime.
On top of this the fluid and the scalar field are a source of metric perturbations.
In the synchronous gauge these produce a change in the space-time interval
\begin{equation}
ds^2 = -dt^2 + (\delta_{ij} + h_{ij}) dx^i dx^j.
\end{equation}
The metric perturbations $h_{ij}$ are sourced by shear stress,
which is the transverse-traceless part of the energy-momentum tensor $\Pi_{ij}$ of eq. \eqref{eq:ep_tensor_general_matrix}.
This results in a wave equation for $h_{ij}$ with a source term,
\begin{equation}
\ddot{h}_{ij} - \nabla^2 h_{ij} = 16 \pi G \Pi_{ij}.
\end{equation}
\todo{Clarify where this comes from}
The energy-momentum tensor $T_{ij}$ consists of contributions from the fluid and the scalar field
\begin{align}
T^f_{ij}    &= (e+p) \gamma^2 v_i v_j + p \delta_{ij} \\
T^\phi_{ij} &= \delta_i \phi \delta_j \phi - \frac{1}{2}(\delta \phi)^2 \delta_{ij}.
\end{align}

Let us introduce the tensor \cite[eq. 1.35]{maggiore_gw_2008}
\begin{equation}
P_{ij}(\bm{k}) = \delta_{ij} - \hat{k}_i \hat{k}_j.
\end{equation}
This tensor is symmetric, transverse ($\hat{k}_i P_{ij}(\bm{k}) = 0$) and a projector ($P_{ik}P_{kj} = P_{ij}$), and its trace is $P_{ii} = 2$.
Using $P_{ij}$ we can construct another tensor that we call Lambda tensor \cite[eq. 1.36]{maggiore_gw_2008}
\begin{equation}
\Lambda_{ij,kl}(\bm{k}) = P_{ik}(\bm{k}) P_{jl}(\bm{k}) - \frac{1}{2} P_{ij}(\bm{k}) P_{kl}(\bm{k}).
\end{equation}
\iffalse
This is still a projector, in the sense that
\begin{equation}
\Lambda_{ij,kl} \Lambda_{kl,mn} = \Lambda_{ij,mn}.
\end{equation}
It is also transverse on all of its indices as $n^i \Lambda_{ij,kl} = 0$ for all indices ($ijkl$}.
It is traceless with respect to the ($i,j$) and ($k,l$) index pairs,
\begin{equation}
\Lambda_{ii,kl} = \Lambda_{ij,kk} = 0.
\end{equation}
\fi
For further details on the Lambda tensor, please see \cite[ch. 1.2]{maggiore_gw_2008}.

The spectral density of the time derivative of the perturbations, $P_{\dot{h}}(\bm{k},t)$, is defined by
\cite[eq. 3.4]{hindmarsh_gw_pt_2019}
\begin{equation}
\langle \dot{h}_{\bm{k}}^{ij}(t) \dot{h}_{\bm{k}}^{ij}(t) \rangle = P_{\dot{h}}(\bm{k},t) (2\pi)^3 \delta (\bm{k} + \bm{k}').
\label{eq:hbracket}
\end{equation}
\todo{Should the second $k$ be $k'$?}

% Defining Omega_gw
The energy density of gravitational waves is given by
\cites[eq. 3.3]{hindmarsh_gw_pt_2019}[eq. 1.135, 7.193]{maggiore_gw_2008}
\begin{align}
e_{gw}
&= \frac{1}{32 \pi G} \int dk^3 \langle \dot{h}_{ij} \dot{h}_{ij} \rangle \\
&= \frac{1}{32 \pi G} \int dk \frac{k^2}{2 \pi^2} P_{\dot{h}}(k).
\end{align}
In many textbooks and articles the integral in the first expression is implicit.
The factor of $\frac{k^2}{2\pi^2}$ results from the conversion to spherical coordinates.
The critical energy density corresponding to a flat universe is given by
\cite[eq. 7.196]{maggiore_gw_2008}
\begin{equation}
e_c = \frac{3 H^2}{8 \pi G}.
\label{eq:e_crit}
\end{equation}
Let us define the dimensionless quantity
\begin{equation}
\Omega_{gw} \equiv \frac{e_{gw}}{e_c}.
\label{eq:omega_gw}
\end{equation}
This will be needed later in eq. \eqref{eq:gw_pow_spec}.

The gravitational wave power spectrum is defined by
\cite[eq. 3.45]{hindmarsh_gw_pt_2019}
\todo{Find a better source}
\begin{equation}
\mathcal{P}_{gw}(k)
\equiv \frac{d \Omega_{gw}}{d \ln (k)}
% = \frac{1}{\bar{\rho}} \frac{1}{32 \pi G} \mathcal{P}_{\dot{h}}(k)
= \frac{1}{12 H^2} \frac{k^3}{2\pi} P_{\dot{h}}(k)
= \frac{1}{12 H^2} \mathcal{P}_{\dot{h}}(k),
\label{eq:gw_pow_spec}
\end{equation}
where the first step comes directly using the equations above.
It should be noted that in \cite[eq. 3.6]{hindmarsh_gw_pt_2019} it is presumed that $\bar{\rho}=e_c$.
We will now proceed to define quite a few similar quantities.
To avoid confusion, their symbols and names are listed in table \ref{table:symbols}.


The spectral density of the velocity power spectrum is given by
\begin{equation}
P_{\tilde{v}} (q) = 2 P_v (q).
\end{equation}
A spectral density can be converted to a power spectrum with
\begin{equation}
\mathcal{P}(k) = \textcolor{gray}{(2)} \frac{k^3}{2 \pi^2} P(k).
\label{eq:pow_spec}
\end{equation}
The factor of two is due to the fact that the velocity Fourier transform includes waves moving in both directions.
Whether it's included or not is dependent on the context.
Therefore, the velocity power spectrum is given by eq. \eqref{eq:pow_spec} \cite[eq. 4.18]{hindmarsh_gw_pt_2019}
\begin{equation}
\mathcal{P}_{\tilde{v}} (q) = 2 \frac{q^3}{2\pi^2} P_{\tilde{v}} (q)
\end{equation}

There exists a solution to the gravitational wave equation in the form of
\begin{equation}
h_{ij} (\bm{k},t) = (16 \pi G) \Lambda_{ij,kl}(\bm{k}) \int_0^t dt' \frac{\sin [k(t-t')]}{k} T_{kl}(\bm{k},t')
\end{equation}
It is sufficient to replace $T_{ij}$ with the tensor
\begin{equation}
\tau_{ij} = \gamma^2 w v_i v_j + \delta_ \phi \delta_j \phi
\end{equation}
\todo{This is $T_{ij}$ with the diagonal elements removed. TT gauge?}
Therefore we have
\begin{multline}
\langle \dot{h}_{\bm{k}_1}^{ij}(t) \dot{h}_{\bm{k}_2}^{ij}(t) \rangle \\
= (16 \pi G)^2 \int_0^t dt_1 dt_2 \cos [k_1(t-t_1)] \cos [k_2(t-t_2)] \Lambda_{ij,kl}(\bm{k})
\langle \tau_{\bm{k}_1}^{ij}(t_1) \tau_{\bm{k}_2}^{kl}(t_2) \rangle
\end{multline}
Let us define the unequal time correlator (UETC) of the fluid shear stress $U_\Pi$ with
\begin{equation}
\Lambda_{ij,kl}(\bm{k}) \langle \tau_{\bm{k}_1}^{ij}(t_1) \tau_{\bm{k}_2}^{kl}(t_2) \rangle
= U_\Pi (k_1, t_1, t_2) (2 \pi)^3 \delta(\bm{k}_1 + \bm{k}_2).
\label{eq:hbracket2}
\end{equation}
Using this we can make a simple substitutions to eq. \eqref{eq:hbracket} and \eqref{eq:hbracket2},
resulting in an expression for the spectral density as
\begin{equation}
P_{\dot{h}} (k,t) = (16 \pi G)^2 \int_0^t dt_1 \int_0^t dt_2 \cos [k(t-t_1)] \cos [k(t-t_2)] U_\Pi (k, t_1, t_2).
\end{equation}


Spectral density of the plane wave components of the velocity field \cite[eq. 4.17]{hindmarsh_gw_pt_2019}
\begin{equation}
P_v(q) = \frac{1}{\beta^6}{R_*^3} \int d\tilde{T} \nu(\tilde{T}) \tilde{T}^6 |A(\frac{\tilde{T}q}{\beta})|^2.
\end{equation}


\begin{table}
\caption{Spectral quantities}
\begin{tabular}{l|l}
$P_{\dot{h}}$ & Spectral density of $\dot{h}$ \\
$P_v$ & Spectral density of the velocity field \\
$P_{\tilde{v}}$ & Spectral density of the Fourier transformation of the velocity field \\
$\tilde{P}_v$ & Scaled velocity spectral density \\
$\tilde{P}_{gw}$ & Spectral density of the gravitational wave power spectrum \\
$\mathcal{P}_v$ & ??? \\
$\mathcal{P}_{\tilde{v}}$ & Velocity power spectrum \\
$\mathcal{P}_gw$ & Gravitational wave power spectrum
\end{tabular}
\label{table:symbols}
\end{table}

Let us define
\begin{align}
\tilde{q} &= \frac{k}{c_s} - q, \\
q_\pm &= \frac{k(1 \pm c_s)}{2 c_s}, \\
\mu &= \frac{2qc_s - k(1 - c_s^2)}{2qc_s^2}.
\end{align}
Using these the the growth rate of the spectral density of $\dot{h}$ can be written as
\begin{equation}
\dot{P}_{\dot{h}} (k) = (16 \pi G)^2 \frac{\bar{w}^2}{4 \pi k c_s} \int_{q_-}^{q_+} \frac{q^3}{\tilde{q}} (1 - \mu^2)^2 P_v(q) P_v(\tilde{q}).
\end{equation}

Let us introduce the scaled velocity spectral density $\tilde{P}_v$ as
\begin{equation}
\tilde{P}_v(qL_f) \equiv \frac{P_v(q)}{L_f^3} \bar{U}_f^2,
\end{equation}
where $L_f$ is a length scale in the velocity field, and $\bar{U}_f$ is the RMS fluid velocity.
Let us also define two additional quantities
\begin{equation}
y = kL_f, \quad z = qL_f, \quad z_\pm = y \frac{1 \pm c_s}{2 c_s}.
\label{eq:gw_yz}
\end{equation}
From these we get the intermediate results
\begin{align}
\tilde{q}L_f &= \frac{y}{c_s} - z, \\
\frac{y}{c_s} - z &= z_+ + z_- - z, \\
1 - \mu^2 &= \frac{a}{z^2}(z-z_+)(z-z_-)
\label{eq:gw_mu_z}
\end{align}
Therefore the asymptotic growth rate of the spectral density can be written as
\begin{equation}
\dot{P}_{\dot{h}}(y) =
\left( 16 \pi G \bar{w} \bar{U}_f^2 \right)^2
\frac{L_f^4}{4 \pi y c_s}
\int_{z_-}^{z_+} dz
\frac{z^3}{\frac{y}{c_s} - z}
(1 - \mu^2)^2
\tilde{P}_v (z) \tilde{P}_v (\frac{y}{c_s} - z).
\end{equation}
The growth rate of the gravitational wave spectrum relative to the Hubble rate is defined as and using \eqref{eq:gw_pow_spec} can be expressed as
\cite[eq. 3.46]{hindmarsh_gw_pt_2019}
\begin{equation}
\mathcal{P}'_{\text{gw}} \equiv \frac{1}{H} \frac{d}{dt} \mathcal{P}_{\text{gw}}
= 3 \left( \Gamma \bar{U}_f^2 \right)^2 (HL_f) \frac{(kL_f)^3}{2 \pi^2} \tilde{P}_{\text{gw}} (kL_f),
\end{equation}
where using \eqref{eq:gw_mu_z} we have $\tilde{P}_{\text{gw}}$,the dimensionless spectral density function
\cite[eq. 3.47]{hindmarsh_gw_pt_2019}
\footnote{In \cite{hindmars_gw_pt_2019} the $\tilde{P}_v$ is denoted as $\bar{P}_v$ by mistake.}
\begin{equation}
\tilde{P}_\text{gw} (y) = \frac{1}{4\pi yc_s} \left(\frac{1-c_s^2}{c_s^2}\right)^2
\int_{z_-}^{z_+} \frac{dz}{z}
\frac{(z-z_+)^2(z-z_-)^2}{z_+ + z_- - z}
\tilde{P}_v (z) \tilde{P}_v (z_+ + z_- z).
\end{equation}
To get the total gravitational wave power spectrum for a stationary velocity power spectrum with a lifetime $\tau_v$ is given by
\begin{equation}
\mathcal{P}_\text{gw}(k)
= \int_{t_0}^{t} H \mathcal{P}'_\text{gw}
= 3 \left( \Gamma \bar{U}_f^2 \right)^2 (H \tau_v)(H L_f) \frac{(kL_f)^3}{2\pi^2} \tilde{P}_\text{gw} (kL_f).
\end{equation}
This is the equation used by PTtools to compute the gravitational wave spectrum.
\todo{Check if this is really "the" equation}



\cites{hindmarsh_gw_pt_2019}[ch. 8]{lecture_notes}
