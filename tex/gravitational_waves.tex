Energy density of gravitational waves is given by
\begin{equation}
e_{gw} = \frac{1}{32 \pi G} \langle \dot{h}_{ij}^{TT} \dot{h}_{ij}^{TT} \rangle.
\end{equation}
\cites[eq. 3]{hindmarsh_gw_pt_2019}[eq. 1.135, 7.193]{maggiore_gw_2008}
The critical energy density corresponding to a flat universe is given by
\begin{equation}
e_c = \frac{3 H^2}{8 \pi G}.
\label{eq:e_crit}
\end{equation}
Let us define the dimensionless quantity
\begin{equation}
\Omega_{gw} \equiv \frac{\rho_{gw}}{\rho_c}.
\end{equation}
The spectral density of the time derivative of the perturbations $P_{\dot{h}}$ is defined by
\begin{equation}
e_{gw} = \frac{1}{32 \pi G} \int dk \frac{k^2}{2 \pi^2} P_{\dot{h}}(k).
\end{equation}
\todo{Why a factor of $k^2$ and not $k$?}
The gravitational wave power spectrum is given by
\begin{equation}
\mathcal{P}_{gw}(k)
\equiv \frac{d \Omega_{gw}}{d \ln (k)}
= \frac{1}{\bar{\rho}} \frac{1}{32 \pi G} \mathcal{P}_{\dot{h}}(k)
= \frac{1}{12 H^2} \mathcal{P}_{\dot{h}}(k),
\end{equation}
where in the first step we have used eq. TODO and in the second step eq. \eqref{eq:e_crit}.
\todo{How to define $\mathcal{P}_{\dot{h}}(k)$?}

A spectral density can be converted to a power spectrum with
\begin{equation}
\mathcal{P}(k) = 2 \frac{k^3}{2 \pi^2} P(k).
\label{eq:pow_spec}
\end{equation}
The factor of two is due to the fact that the velocity Fourier transform includes waves moving in both directions.

Spectral density of the plane wave components of the velocity field \cite[eq. 4.17]{hindmarsh_gw_pt_2019}
\begin{equation}
P_v(q) = \frac{1}{\beta^6}{R_*^3} \int d\tilde{T} \nu(\tilde{T}) \tilde{T}^6 |A(\frac{\tilde{T}q}{\beta})|^2.
\end{equation}
The velocity power spectrum is given by eq. \eqref{eq:pow_spec} \cite[eq. 4.18]{hindmarsh_gw_pt_2019}
\begin{equation}
\mathcal{P}_{\tilde{v}} (q) = 2 \frac{q^3}{2\pi^2} P_v(q)
\end{equation}

Spectral density of the gravitational waves \cite[eq. 3.47]{hindmarsh_gw_pt_2019}
\begin{equation}
\tilde{P}_{gw}(y)
= \frac{1}{4 \pi y c_s}
\left( \frac{1 - c_s^2}{c_s^2} \right)^2
\int_{z_-}^{z_+} \frac{dz}{z}
\frac{(z-z_+)^2(z-z_-)^2}{z_+ + z_- - z}
\tilde{P}_v (z) \tilde{P}_v (z_+ + z_- - z).
\end{equation}

\cites{hindmarsh_gw_pt_2019}[ch. 8]{lecture_notes}
