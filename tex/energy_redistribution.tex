\section{Energy redistribution}
\label{energy_redistribution}
Once the velocity profile of a bubble is known, it can be used to compute various thermodynamic quantities.

We have assumed that the background spacetime is Minkowski, and that the spacetime does not expand significantly during the phase transition.
Therefore, the total energy of the system is conserved and
\begin{equation}
E = 4 \pi \int_0^R dr r^2 T^{00}
\end{equation}
is a constant for $R$ larger than the fluid velocity profile.
It should be noted that this energy conservation applies only to the total energy system,
and not the energy of the fluid, as there is energy transfer from the field to the fluid.
\cite[p. 21]{lecture_notes}
Correspondingly the mean energy density $\bar{e}$ is defined as \cite[p. 39]{lecture_notes}
\begin{equation}
\bar{e} = \frac{3}{\xi_\text{max}^3} \int_0^{\xi_\text{max}} d\xi \xi^2 e = e(w_n, \phi_s).
\label{eq:e_conservation}
\end{equation}
The mean enthalpy density $\bar{w}$ is defined as
\begin{equation}
\bar{w} = \frac{3}{\xi_\text{max}^3} \int_0^{\xi_\text{max}} d\xi \xi^2 w
\label{eq:wbar}
\end{equation}
Using these we can define the energy fluctuation variable
\begin{equation}
\lambda(x) = \frac{e(x) - \bar{e}}{\bar{w}}.
\label{eq:lambda}
\end{equation}

For an ideal fluid of eq. \eqref{eq:ep_tensor} \cite[eq. B.23]{hindmarsh_gw_pt_2019}
% This is easy to prove by hand.
\begin{equation}
T^{00} = w\gamma^2 - p = w\gamma^2 v^2 + e = w\gamma^2 v^2 + \frac{3}{4}w + \theta.
\end{equation}
Inserting this to the energy conservation equation \eqref{eq:e_conservation} results in the equation \cite[eq. B.24]{hindmarsh_gw_pt_2019}
\begin{equation}
e_K + \Delta e_Q = - \Delta e_\theta,
\label{eq:energy_components}
\end{equation}
which consists of the terms defined below.
The kinetic energy density is given by
\begin{equation}
e_K \equiv 4 \pi \int_0^{\xi_\text{max}} d\xi \xi^2 w \gamma^2 v^2.
\label{eq:kinetic_energy_density}
\end{equation}
It denotes the energy that is converted to macroscopic fluid movement.
The thermal energy density difference is given by
\begin{equation}
\Delta e_Q \equiv 4 \pi \int_0^{\xi_\text{max}} d\xi \xi^2 \frac{3}{4} (w - w_n).
\label{eq:thermal_energy_density}
\end{equation}
It denotes the energy that is converted to microscopic fluid movement.
The trace anomaly difference is given by
\begin{equation}
\Delta e_\theta \equiv 4 \pi \int_0^{\xi_\text{max}} d\xi \xi^2 (\theta - \theta_n).
\end{equation}
It can be considered as the potential energy available for transformation.
We can obtain the volume-average entropy density difference similarly as the energy density, with
\begin{equation}
\Delta s_\text{avg} = 4\pi \int_0^{\xi_\text{max}} d\xi \xi^2 \left( s(w,\phi) - s(w_n, \phi_s) \right).
\end{equation}
The upper integration limit $\xi_\text{max}$ can be chosen arbitrarily as long as it's outside the bubble, as outside the bubble all three quantities go to zero outside the fluid shell due to $v=0$, $w=w_n$ and $\theta = \theta_n$.
A convenient choice is $\xi_\text{max} = \max (v_w, \xi_{sh})$.
\cite[eq. B.25]{hindmarsh_gw_pt_2019}

It should be noted that the trace anomaly is not quite equivalent to the thermal potential energy density $V_T(T,\phi)$, as using eq. \eqref{eq:enthalpy_pressure}, \eqref{eq:enthalpy_sum} and \eqref{eq:p_general} we can see that for the case of temperature-independent $g$,
\begin{equation}
\theta = V_T - \frac{1}{4} T \frac{\partial V_T}{\partial T}.
\end{equation}
Therefore, not all of the potential energy difference can be turned into kinetic and thermal energy.
\cite[ch. B.2]{hindmarsh_gw_pt_2019}

The fraction of total energy that is converted to kinetic energy is known as the kinetic energy fraction,
\begin{equation}
K \equiv \frac{e_K}{\bar{e}}.
\label{eq:kinetic_energy_fraction}
\end{equation}

Kinetic and thermal efficiency factors quantify the fraction of the available energy $\Delta e_\theta$ that is converted to kinetic and thermal energy.
They can be defined as
\begin{equation}
\kappa \equiv \frac{e_K}{| \Delta e_\theta |}, \quad
\omega \equiv \frac{\Delta e_Q}{\Delta e_\theta}.
\label{eq:kappa_omega}
\end{equation}
Since energy is conserved by eq. \eqref{eq:energy_components},
these have have the property that $\kappa + \omega = 1$.

It should be noted that some sources such as \cites[eq. 36]{giese_2020}[eq. 12, 14]{giese_2021}
use different definitions for $\kappa$:
\begin{align}
\kappa_{\bar{\Theta}_+} &\equiv \frac{4 e_K}{D \bar{\Theta}(T_+)} = \frac{4 e_K}{3 \alpha_{\bar{\Theta}_+} w_+},
\label{eq:kappa_thetabar_plus} \\
\kappa_{\bar{\Theta}_n} &\equiv \frac{4 e_K}{D \bar{\Theta}(T_n)} = \frac{4 e_K}{3 \alpha_{\bar{\Theta}_n} w_n}.
\label{eq:kappa_thetabar_n}
\end{align}

The enthalpy-weighted mean square fluid 4-velocity around the bubble is defined by
\begin{equation}
\bar{U}_f^2 = \frac{\Delta e_Q}{w_n}.
\label{eq:ubarf2}
\end{equation}
The adiabatic index is defined as
\begin{equation}
\gamma \equiv \frac{c_P}{c_V},
\end{equation}
where $c_P$ is the specific heat capacity at constant pressure,
and $c_V$ is the specific heat capacity at constant volume.
For a classical monoatomic fluid $\gamma = \frac{5}{3}$,
and for an ultrarelativistic fluid $\gamma = \frac{4}{3}$.
The enthalpy-weighted mean square fluid 4-velocity around the bubble and the kinetic energy density fraction of eq. \eqref{eq:kinetic_energy_fraction} are linked by
\begin{equation}
K = \Gamma \bar{U}_f^2,
\label{eq:kinetic_energy_fraction2}
\end{equation}
where
\begin{equation}
\Gamma \equiv \frac{\bar{w}}{\bar{e}}.
\label{eq:mean_adiabatic_index}
\end{equation}
In the case of the ultrarelativistic bag model $\Gamma$ is the mean adiabatic index,
but for a more general model this may not be the case.
