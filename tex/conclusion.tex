\iffalse
\begin{itemize}
    \item Briefly discuss the results
    \item Highlight alternative research questions
    \item Put the work in context
    \item etc.
\end{itemize}

"A good thesis should bite itself in the tail"
\fi

In this thesis PTtools has been expanded from a compact simulation script based on the bag model
to a comprehensive framework for simulating first order phase transitions and their gravitational wave spectra using the Sound Shell Model,
including the conversion to the gravitational spectra today.

PTtools was tested using the constant sound speed model for simplicity,
but the interface between the model and the rest of PTtools is general in such a way
that it supports arbitrary particle physics models as long as they provide $V_s(T,\phi), V_b(T,\phi)$ and two of $g_p(T,\phi), g_e(T,\phi), g_s(T,\phi)$ or two of $p(T,\phi), e(T,\phi), s(T,\phi)$, but preferably all three for numerical precision.
This enables the easy integration of models developed by other researchers,
and therefore the comparison of the gravitational wave spectra of these models.
This is also to the author's best knowledge the first time that phase transitions in the early universe can be simulated with temperature-dependent speeds of sound without resorting to time-consuming 3D simulations.
\todo{Is this true?}

The numerical precision of the simulation is a challenge for thin hybrid shells,
but this is a small section of the $(v_\text{wall}, \alpha_n)$ parameter space,
and for the vast majority of the parameter combinations the simulator works reliably.

This master's thesis lays the groundwork for the author's PhD thesis on the topic.
Now there is a framework for determining the gravitational wave spectrum from the phase transition parameters, including temperature-dependent speed of sound.
Determining the parameters of a phase transition from LISA data will require solving the inverse problem: what are the phase transition parameters based on the gravitational wave spectrum?
This will require novel tools such as machine learning or Markov chain Monte Carlo simulations.
There is also potential for integrating PTtools with the existing web-based simulation utility PTPlot to create a comprehensive but easy-to-use utility for researchers to simulate phase transitions with various parameters.

PTtools will be published as open source soon after the publication of this thesis on GitHub at \cite{pttools}.
This will enable other researchers to use the library and to integrate their models to it.
