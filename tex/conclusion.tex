\iffalse
\begin{itemize}
    \item Briefly discuss the results
    \item Highlight alternative research questions
    \item Put the work in context
    \item etc.
\end{itemize}

"A good thesis should bite itself in the tail"
\fi

In this thesis PTtools has been expanded from a compact simulation script based on the bag model
to a comprehensive framework for simulating the fluid velocity profiles of first-order phase transitions in the early universe
and their gravitational wave spectra based on the Sound Shell Model,
including the conversion to the gravitational spectra today.

PTtools was tested with the constant sound speed model due to the availability of reference data by Giese et al. \cite{giese_2021},
and it was shown to work reliably for a broad range of the combinations of the sound speeds $c_{s,s}$ and $c_{s,b}$, the phase transition strength $\alpha_n$ and the wall speed $v_\text{wall}$.
The PTtools fluid velocity profile solver has been updated beyond that of the available references in such a way
that it supports arbitrary particle physics models with a temperature-dependent sound speed $c_s(T,\phi)$,
as long as they provide $V_s(T,\phi), V_b(T,\phi)$ and two of $g_p(T,\phi), g_e(T,\phi), g_s(T,\phi)$ or two of $p(T,\phi), e(T,\phi), s(T,\phi)$, but preferably all three for numerical precision.
This enables the easy integration of models developed by other researchers,
and therefore the comparison of the gravitational wave spectra of these models.
This is also to the author's best knowledge the first time that phase transitions in the early universe have been simulated with temperature-dependent speeds of sound without resorting to time-consuming 3D simulations.

The numerical results of the fluid shell solver have been demonstrated to be consistent with a reference
for a broad range of parameters.
However, some special cases are still challenging.
The most notable of these are very thin hybrid shells near the Chapman-Jouguet speed $v_\text{wall} = v_{CJ}$,
where the thickness of the fluid shell in front of the wall is comparable to the minimum step possible for the ODE solvers.
This limits the precision of the overall solver in these special cases.
However, these special cases compose a very small section of the overall parameter space,
and for the vast majority of the parameter space,
the PTtools solver produces reliable results.

The performance of PTtools is also notable.
The creation of the gravitational wave spectrum today from the phase transition parameters takes less than TODO per bubble on a laptop.
This is in a stark contrast with the 3D hydrodynamic simulations that require the computational power of supercomputers.
This demonstrates that the Sound Shell Model is an essential tool in investigating
the effects of the phase transition parameters on the resulting gravitational wave spectrum.

This master's thesis lays the groundwork for the author's PhD thesis on the topic.
Now there is a framework for determining the gravitational wave spectrum from the phase transition parameters, including temperature-dependent speed of sound.
Determining the parameters of a phase transition from LISA data will require solving the inverse problem: what are the phase transition parameters based on the gravitational wave spectrum?
This will require novel tools such as machine learning or Markov chain Monte Carlo simulations.
There is also potential for integrating PTtools with the existing web-based simulation utility PTPlot to create a comprehensive but easy-to-use utility for researchers to simulate phase transitions with various parameters.
These topics will be investigated in the author's PhD studies.

PTtools will be published as open source soon after the publication of this thesis on GitHub at \cite{pttools}.
This will enable the research community to integrate their various particle physics models,
and to simulate the resulting gravitational wave spectra.
This will provide the LISA simulation pipeline with various waveforms of interest,
and if one of them is eventually found in LISA data,
this will result in a groundbreaking discovery
that will point the direction for the development of particle physics beyond the Standard Model.
