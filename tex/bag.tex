\section{The bag model}
\label{bag_model}
For phase transitions in the early universe,
the most commonly used equation of state is the bag model, for which
\begin{equation}
g_{e\pm} = g_{p\pm} = g_{s\pm} = \frac{90}{\pi^2} a_\pm,
\end{equation}
where $a_\pm$ are constants with $a_+ > a_-$.
$+$ refers to the symmetric phase, and $-$ refers to the broken phase.
The potentials $V_\pm$ are also constants with $V_+ > V_-$.
This results in the bag equation of state
\cites[eq. 7.33]{lecture_notes}[eq. 8-9]{giese_2020}
\begin{equation}
p_\pm = a_\pm T^4 - V_\pm.
\label{eq:bag_p}
\end{equation}
The potential difference $\Delta V \equiv V_+ - V_-$, also known as $\epsilon$,
is the temperature-independent vacuum energy that is released in the phase transition.
The potentials are usually shifted so that $V_b = 0$.
The name of the bag model originates from quantum chromodynamics (QCD),
where it's used to describe the proton as a bag of quarks.
\cite{giese_2020}

Using eq.~\eqref{eq:entropy_density} we get the entropy density
\begin{equation}
s_\pm = 4 a_\pm T^3,
\end{equation}
and with eq.~\eqref{eq:enthalpy_entropy} the enthalpy density
\begin{equation}
w_\pm = 4 a_\pm T^4.
\end{equation}
Finally with eq.~\eqref{eq:enthalpy_sum} we get the energy density
\begin{equation}
e_\pm = 3 a_\pm T^4 + V_\pm.
\end{equation}
The speed of sound from eq.~\eqref{eq:cs2_explicit} simplifies to
\begin{equation}
c_s^2 = \frac{dp}{de} = \frac{dp/dT}{de/dT} = \frac{1}{3}.
\end{equation}
As in section~\ref{general_eq}, this corresponds to assuming both of the phases to be ultrarelativistic.
It also happens to be the same as the $\tilde{v}_-$ that corresponds to eq.~\eqref{eq:v_tilde_plus_limit},
and therefore the Chapman-Jouguet speed $v_{CJ}$ of the bag model is given by eq.~\eqref{eq:v_tilde_plus_limit}.
The trace anomaly of eq.~\eqref{eq:theta} simplifies to
\begin{align}
\theta_\pm = V_\pm.
\end{align}
Therefore, the transition strength of~\eqref{eq:alpha_plus} simplifies to
\begin{equation}
\alpha_{+,\text{bag}} = \frac{4 \Delta V}{3 w_+},
\label{eq:alpha_plus_bag}
\end{equation}
and the transition strength at nucleation temperature from~\eqref{eq:alpha_n} simplifies to
\begin{equation}
\alpha_{n,\text{bag}} = \frac{4 \Delta V}{3 w_n}.
\label{eq:alpha_n_bag}
\end{equation}
This is trivial to invert, giving enthalpy at the nucleation temperature $w_n$.
