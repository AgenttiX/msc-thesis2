\section{Types of solutions}
Solutions to the hydrodynamic equations~\eqref{eq:hydro_param1},~\eqref{eq:hydro_param2} are obtained by integrating away from the position of the bubble wall, $\xi_w$.
There are two relevant external boundary conditions.
To maintain spherical symmetry
\begin{equation}
\lim_{\xi \rightarrow 0} v = 0.
\end{equation}
To maintain causality
\begin{equation}
\lim_{\xi \rightarrow 1} v = 0,
\end{equation}
as no information can propagate faster than light, and we assume the fluid to be stationary until a signal from the expanding bubble arrives.
In addition to these there is one boundary condition for each side of the wall as
\begin{equation}
\lim_{\xi \rightarrow \xi_w^\pm = v_w \pm \delta, \delta \rightarrow 0} v = v_\pm = \mu (\xi_w, \tilde{v}_\pm),
\end{equation}

There are two ways to satisfy these conditions.
We can either start at $v=0$, or from a region where $\xi > c_{s-}(w)$ and $\mu(\xi,v) > c_{s-}(w)$, and therefore $\frac{dv}{d\xi} > 0$,
and integrating backwards in $\xi$ over a route where $\mu(\xi,v) > c_{s-}(w)$ is satisfied.
The other way to reach $v=0$ is by a discontinuity, i.e. a shock.
This leads to two classes of solutions.
Another way to see that there are two classes of solutions is by noting
that the equations \eqref{eq:v_tilde_plus} and \eqref{eq:v_tilde_minus} have two branches.
These solutions are classified in table \ref{table:solution_types}.

The solutions with $\tilde{v}_+ > \tilde{v}_-$ are known as detonations.
% In this case there is no way for the front to influence the fluid ahead,
% and the outermost front is a shock.
Detonations are further characterised by the scale of their $\tilde{v}_-$.
In weak detonations $\tilde{v}_- > c_{s-}(w_-)$.
However, weak detonations are not possible in an exothermic reaction such as a cosmological phase transition.
\cite[p. 265]{rezzolla_relativistic_2013}
\todo{According to Hindmarsh weak detonations could exist. Check whether this is possible!}
\todo{How should we modify the $(\xi,v)$ plot for weak detonations to appear?}
Correspondingly, detonations with $\tilde{v}_- < c_{s-}(w_-)$ are known as strong detonations.
However, they are unstable and will naturally evolve into the third class of detonations,
the Chapman-Jouguet detonations, for which $\tilde{v}_- = c_{s-}(w_-)$.
\cite[p. 279]{rezzolla_relativistic_2013}
Since we are investigating a self-similar bubble that has already evolved for some time,
all our detonations of interest are Chapman-Jouguet detonations.
\todo{What would a strong detonation look like in the $(\xi,v)$ plot?}
In Chapman-Jouguet detonations the shock and phase boundary fronts are unified to a single front.

The solutions where $\tilde{v}_+ < \tilde{v}_-$ are known as deflagrations.
In them the phase front can influence the fluid ahead, and the wall is preceded by an accelerating fluid and a shock.
Like with weak detonations, strong deflagrations are not possible in an exothermic reaction.
\cite[p. 267]{rezzolla_relativistic_2013}
Therefore only weak and Chapman-Jouguet deflagrations are possible.
In weak detonations the fluid inside the phase boundary is still, and the preceding shock is weak and
known as the precompression front.

In several articles \cites[p. 37]{lecture_notes}[p. 35]{hindmarsh_gw_pt_2019}
Chapman-Jouguet detonations are known as supersonic deflagrations or hybrids,
\todo{Chapman-Jouguet detonations are the fastest, but the relation between hybrids and C-J is not one-to-one. Fix this}
as in them the wall speed exceeds the speed of sound in the broken phase $c_{s-}$,
and the fluid is moving inside the phase boundary as well, as in a detonation.
\todo{Investigate what a strong detonation looks like.}

\begin{table}[ht!]
\small
\begin{tabular}{r|c|c}
                & Detonations            & Deflagrations \\
                & $p_+ < p_-, \tilde{v}_+ > \tilde{v}_-$ & $p_+ > p_-, \tilde{v}_+ < \tilde{v}_-$ \\ \hline
Weak            & {\color{gray} $\tilde{v}_+ > c_{s+}(w_+), \ \tilde{v}_- > c_{s-}(w_-)$} & $\tilde{v}_+ < c_{s+}(w_+), \ \tilde{v}_- < c_{s-}(w_-)$ \\
Chapman-Jouguet & $\tilde{v}_+ > c_{s+}(w_+), \ \tilde{v}_- = c_{s-}(w_-)$ & $\tilde{v}_+ < c_{s+}(w_+), \ \tilde{v}_- = c_{s-}(w_-)$ \\
Strong          & {\color{gray} $\tilde{v}_+ > c_{s+}(w_+), \ \tilde{v}_- < c_{s-}(w_-)$} & {\color{gray} $\tilde{v}_+ < c_{s+}(w_+), \ \tilde{v}_- > c_{s-}(w_-)$} \\
\end{tabular}
\label{table:solution_types}
\end{table}

\begin{figure}[h!]
\centering
\missingfigure{TODO}
\caption{Three different types of relativistic combustion \cite[fig. 14]{lecture_notes}}
\label{fig:solution_types}
\end{figure}


\clearpage
\FloatBarrier
\subsection{Speed limits}
The observables of a physical system must be real, including $\tilde{v}_+$ and $\tilde{v}_-$.
Therefore the expressions in the square roots of \eqref{eq:v_tilde_plus} and \eqref{eq:v_tilde_minus} must be real.
By simplifying the expression in the square root of \eqref{eq:v_tilde_plus} we see that the square root is real $\forall \alpha_+ \geq 0$,
but setting the square root in \eqref{eq:v_tilde_minus} to zero gives a limit for $\tilde{v}_+$ as
\begin{equation}
\tilde{v}_+ = \frac{1}{\sqrt{3}} \left( \frac{1 \pm \sqrt{2 \alpha_+ + 3 \alpha_+^2}}{1 + \alpha_+} \right).
\label{eq:v_tilde_plus_limit}
\end{equation}
The positive sign is the lower limit for detonations, and the negative sign is the upper limit for deflagrations.
\footnote{This is the same equation as in \cites[eq. 7.34]{lecture_notes}[eq. B.19]{hindmarsh_gw_pt_2019},
but in those articles there is a typo due to which a factor of 2 is missing from the expression.}

Inserting this value to the equation \eqref{eq:v_tilde_minus} and simplifying the expression with the knowledge that the square root is now zero
we get that
\begin{equation}
\tilde{v}_- = \frac{1}{\sqrt{3}}.
\label{eq:v_tilde_minus_limit}
\end{equation}
This is the lower limit for detonations, and the upper limit for deflagrations.

The Chapman-Jouguet speed is defined as
\begin{equation}
\tilde{v}_+=v_{CJ} \Leftrightarrow \tilde{v}_- = c_{s-}(w_-).
\label{eq:chapman_jouguet}
\end{equation}
Starting from section \eqref{bag_model} we will see that for some models the speed limit set by the Chapman-Jouguet speed of \eqref{eq:chapman_jouguet}
and the condition of \eqref{eq:v_tilde_minus_limit} that the observables are real are equivalent,
but in general this is not the case.

Now we have the necessary knowledge to classify the different regions of fig. \eqref{fig:vplus_vminus}.
If the speeds of sound are equivalent to $\frac{1}{\sqrt{3}}$,
we can have only weak and Chapman-Jouguet detonations and deflagrations.
However, the weak detonations are ruled out for the aforementioned reason that they are not possible in an exothermic reaction.

When the speed of sound in the broken phase $c_{s-}(w_-) < \frac{1}{\sqrt{3}}$,
we can have a strong deflagration where $c_{s-}(w_-) < \tilde{v}_- < \frac{1}{\sqrt{3}}$.
Correspondingly, when $c_{s-}(w_-) > \frac{1}{\sqrt{3}}$,
we can have a strong detonation where $\frac{1}{\sqrt{3}} < \tilde{v}_-(w_-, \phi_-) < c_{s-}(w_-)$.
However, as previously discussed, a strong detonation is unstable and will evolve into a Chapman-Jouguet detonation.
Since for Chapman-Jouguet detonations $v_w = \tilde{v}_+ > \tilde{v}_-$,
the Chapman-Jouguet speed is the lower speed limit for detonations that have been evolving for a sufficiently long time.
