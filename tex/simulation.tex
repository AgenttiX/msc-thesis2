The equations for the fluid dynamics and gravitational wave production have no analytical solutions in the general case.
Therefore, numerical solutions are required.
To create these solutions, Hindmarsh et al. have developed the phase transition simulation software PTtools~\cite{pttools}.
In this thesis PTtools has been extended to account for speeds of sound that differ from the bag model of section~\ref{bag_model}.
This has enabled PTtools to simulate bubbles using models beyond the bag model,
such as the constant sound speed model of section~\ref{const_cs}.
The previous fluid profile solver relied heavily on analytical shortcuts specific to the bag model,
and the surrounding higher-level functionality was built with this assumption in mind.
Therefore, enabling the support for more complex models required a nearly complete rewrite and significant extension of the entire simulation software.
When also counting the speedup optimisations done by the author as preparations for this update,
these changes resulted in PTtools growing more than an order of magnitude in terms of the lines of code.


\section{Overview of PTtools}
PTtools~\cite{pttools} is a simulation software for modeling the velocity and enthalpy profile of the fluid shell of a single bubble,
and for converting the profile to the velocity spectrum and gravitational wave spectrum using the sound shell model.
PTtools is a
\href{https://www.python.org/}{Python}
library, which uses
\href{https://numpy.org/}{Numpy}
and
\href{https://scipy.org/}{SciPy}
for the numerical simulations,
\href{https://numba.pydata.org/}{Numba}
for speeding up the computations and
\href{https://matplotlib.org/}{Matplotlib}
and
\href{https://plotly.com/}{Plotly}
for plotting.
For PTools and its full documentation, please see~\cite{pttools}.

To use PTtools, the user has to first specify the equation of state.
They can either use the provided bag model (\verb|BagModel|) or constant sound speed model (\verb|ConstCSModel|),
or they can specify their own model by either inheriting from the provided \texttt{Model} class and providing the equation of state analytically.
Another option is to inherit from the provided \verb|ThermoModel| class and provide two of the degrees of freedom $g_p(T,\phi)$ of eq.~\eqref{eq:p_general}, $g_e(T,\phi)$ of~\eqref{eq:e_general} and $g_s(T,\phi)$ of~\eqref{eq:s_general}, and the potential $V(T,\phi)$ of~\eqref{eq:p_general}.
This way the user can let PTtools take care of constructing the equation of state from these parameters.
PTtools then runs various validity checks for the model to ensure that a phase transition will occur and proceed in the system.
For the phase transition to proceed, there has to be a release of energy from the field to the fluid,
and therefore $\Delta V = V_s - V_b \geq 0$.
The model also has to have a critical temperature of eq.~\eqref{eq:critical_temp},
under which it is energetically favourable for the system to transition to the new phase.
As a part of the model initialization, PTtools creates a function for the speed of sound $c_s^2$ of the model.
This function is compiled using Numba to achieve sufficient performance for using this function in the bubble solving.

Once the model has been specified,
the user can create a bubble by creating an instance of the \verb|Bubble| class by providing the model, $v_\text{wall}$ and $\alpha_n$.
PTtools then runs various validity checks for the bubble,
including that there exists a nucleation enthalpy $w_n$ of eq.~\eqref{eq:wn}, a valid solution type of fig.~\ref{fig:solution_types},
and that the nucleation temperature $T_n$ is below the critical temperature.
% LTE = entropy generation, which is OK
% PTtools also warns if the bubble requires deviations from the local thermal equilibrium to exist.
Unless specifically instructed to delay,
PTtools then numerically finds a solution for the hydrodynamic equations of eq.~\eqref{eq:hydro_param1},~\eqref{eq:hydro_param2}, and~\eqref{eq:hydro_param3} and the bubble wall junction conditions of~\eqref{eq:junction_condition_1} and~\eqref{eq:junction_condition_2},
resulting in a fluid velocity profile for the bubble.
This is also known as solving the bubble.
The \verb|Bubble| object has various methods for extracting key quantities such as the thermodynamic quantities of section~\ref{energy_redistribution}.
PTtools also checks the validity of the solution.
% These are explained later:
% including that $\kappa + \omega = 1$,
% that the entropy fluxes at the wall are positive and that their difference is positive,
% and that the total entropy generated is positive.

Further details on PTtools are subject to change as the library is being developed.
Please see the PTtools documentation for the latest information.
An example on the Python code required for creating the gravitational wave power spectrum from the phase transition parameters is below.

\lstset{breaklines=true}
\begin{lstlisting}[language=Python]
import matplotlib.pyplot as plt

from pttools.bubble import Bubble
from pttools.models import ConstCSModel
from pttools.omgw0 import Spectrum

# Specify the equation of state
const_cs = ConstCSModel(
    a_s=1.5, a_b=1, css2=1/3, csb2=1/3-0.01, V_s=1
)

# Create a bubble and solve its fluid profile
bubble = Bubble(const_cs, v_wall=0.5, alpha_n=0.2)
bubble.plot()

# Compute gravitational wave velocity and power spectra for the bubble
spectrum = Spectrum(bubble)
spectrum.plot_multi()

plt.show()
\end{lstlisting}


\section{Bubble solver}
The bubble solver is the numerical simulation that converts the parameters that describe the phase transition, $v_{\text{wall}}$, $\alpha_n$ and the equation of state, to the fluid velocity and enthalpy profiles $v(\xi)$ and $w(\xi)$.
The bubble solver consists of three steps:
1) preparatory steps that provide initial values for a numerical solver,
2) the numerical solver itself, and
3) post-processing to provide output data in a consistent form.
For the bag model the user can also choose to use the previous version of the solver,
which uses several analytical shortcuts based on the assumption that $c_s^2 = \frac{1}{3}$ and is therefore significantly faster.
This solver can be enabled by calling \verb|Bubble(model, v_wall, alpha_n, use_bag_solver=True)|.

The generic solver starts by checking the type of the solution based on the conditions of table~\ref{table:solution_types}.
If the type of the solution cannot be determined automatically, the solver will halt and request the user to provide the type of the solution.
Then the solver will use the bag model to load reference values for $w_+$ and $w_-$ for the given $v_\text{wall}$ and $\alpha_n$.
These reference values will be used as the starting point for the numerical solver.
If the reference values have not been precomputed,
they will be computed and saved to disk at this point.
If there is no reference data for the given parameters,
as is the case for deflagrations with high $\alpha_n$ and low $v_\text{wall}$ for which there is no bag model solution,
then an arbitrary guess of $w_- = 0.3 w_n$ will be used as the starting point for the solver (subject to change),
and the starting guess of the bubble junction condition solver for $w_+$ will be computed based on the bag model equations.
Finally, the Chapman-Jouguet speed of~\eqref{eq:chapman_jouguet} is computed.

Once these preparations have been done,
the solver chooses an algorithm specific to the type of the solution.
Detonations are the simplest case.
Since the fluid is stationary outside the wall,
the junction conditions can be solved directly using
$\tilde{v}_+ = v_\text{wall}$ and $w_+ = w_n$.
Then the solver integrates from $(v_-, w_-)$ to the fixed point at $(\xi=c_{s,b}, v=0)$.
The fluid inside the fixed point is stationary.
There is also another fixed point at $(\xi=1, v=1)$, but we choose the direction of $\tau$ in eq.~\eqref{eq:hydro_param1},~\eqref{eq:hydro_param2} and~\eqref{eq:hydro_param3} so that we integrate in the correct direction.

Subsonic deflagrations are more complicated,
as we cannot compute $v_{-,\text{sh}}, w_{-,\text{sh}}$ from $v_{+,\text{sh}}=0, w_{+,\text{sh}}=w_n$ using the junction conditions,
since we don't know the shock speed beforehand.
For some deflagrations, the shock can also be too small for the numerical accuracy of the junction solver,
since when $v_\text{sh}$ is close to $c_{s,s}$, the fluid profile can vary significantly with small changes of $v_\text{sh}$.
Therefore, we have to start by guessing a $w_-$.
Since the fluid inside the wall is stationary, we know that $\tilde{v}_- = v_{\text{wall}}$.
For a given $(v_-, w_-)$ we can solve the junction conditions, giving $(v_+, w_+)$.
From these we can integrate until we encounter the shock.
However, we cannot directly compute $v_\text{sh}(\xi_\text{sh})$, as it's dependent on $w_{-,\text{sh}}$.
Therefore we have to evaluate $v_\text{sh}(\xi_\text{sh}, w_{-,\text{sh}})$ for each $(\xi, w)$ on the integrated curve to see
when the curve encounters the shock.
This is accomplished by a binary search.
Once we know $\xi_\text{sh}$, we can solve the junction conditions with $v_{+,\text{sh}} = 0$, giving us a $w_{+,\text{sh}}$.
Then we compare this to the given $w_n$.
If they don't match, we adjust our guess for $w_-$ and start again until we have $w_{+,\text{sh}} = w_n$.

Hybrids are the most complicated case, as neither $v_+$ nor $v_-$ is known beforehand.
Therefore we start by guessing a $w_-$, and we know that $\tilde{v}_- = c_s(w_-, \phi_-)$.
Then we can solve the junction conditions to get $(v_+, w_+)$.
With these we can integrate until we encounter the shock.
Then we perform the rest of the steps as for deflagrations, and iterate until we have $w_{+,\text{sh}} = w_n$.
Once this is found, we integrate from our starting point $(v_-, w_-)$ to the fixed point to get the detonation-like tail,
resulting in the full hybrid solution.

For all solution types, once we have the solution,
we add points corresponding to $(\xi=0, w_\text{center})$ and $(\xi=1, w_n)$
so that the solution covers the full range $\xi \in [0, 1]$.
It should be noted, that the resulting solution does not have a fixed step for $\xi$,
and therefore one has to be careful when computing integrals or Fourier transforms of the solution.

The \verb|Bubble| class performs various checks on the solution.
It checks that
1) $\alpha_+(w_+, w_-, \tilde{v}_+)$ can be computed,
2) entropy fluxes across the wall and their difference are non-negative,
3) the total change in the volume-averaged entropy density is non-negative, and
4) that energy is conserved by $\kappa + \omega \approx 1$~\eqref{eq:kappa_omega}.
If any of these checks fail, the solution is marked to have an error.

Once the bubble is solved, the \verb|Bubble| object can be queried for various quantities,
including but not limited to the fluid velocities in both the plasma and wall frames,
the corresponding velocities for the shock,
enthalpies at the wall and at the shock,
various thermodynamic quantities in both bubble volume averaged and volume-averaged forms,
such as the entropy density and entropy fluxes,
kinetic energy density~\eqref{eq:kinetic_energy_density},
kinetic energy fraction~\eqref{eq:kinetic_energy_fraction},
thermal energy density~\eqref{eq:thermal_energy_density},
trace anomaly $\theta$~\eqref{eq:theta},
mean energy density $\bar{e} = e_n$~\eqref{eq:e_conservation},
mean enthalpy density $\bar{w}$~\eqref{eq:wbar},
$\kappa$~\eqref{eq:kappa_omega},
$\kappa_{\bar{\theta}_+}$~\eqref{eq:kappa_thetabar_plus},
$\kappa_{\bar{\theta}_n}$~\eqref{eq:kappa_thetabar_n},
$\omega$~\eqref{eq:kappa_omega},
mean adiabatic index $\Gamma$~\eqref{eq:mean_adiabatic_index} and
$\bar{U}_f^2$~\eqref{eq:ubarf2}.


\section{Spectrum computation}
Once the fluid velocity profile of a bubble is solved as above,
it can be converted to the velocity spectrum and gravitational wave spectrum by constructing a \verb|Spectrum| object.
The spectrum computation is based on the Sound Shell Model~\cite{hindmarsh_gw_pt_2019}.
First, the spectral density of the plane wave components of the velocity field $P_v(q)$ is computed using eq.~\eqref{eq:spec_den_v}.
This is converted to the velocity power spectrum $\mathcal{P}_{\tilde{v}}$ using eq.~\eqref{eq:pow_v}.
The spectral density of the plane wave components of the velocity field $P_v(q)$ is also used to compute the spectral density of gravitational waves $\tilde{P}_{\text{gw}}(y)$ of eq.~\eqref{eq:spectral_density}.
This is converted to the gravitational wave power spectrum $\mathcal{P}_{\text{gw}}$ using~\eqref{eq:gw_pow_spec3}.


\section{Parallel computing}
PTtools provides an interface for creating multiple bubbles and computing quantities from them in parallel on multiple CPU cores,
despite being Python-based software.
This is made possible by the
\href{https://docs.python.org/3/library/multiprocessing.html}{\texttt{multiprocessing}}
module of the Python standard library.
The spectrum computation of a single bubble can also take advantage of multiple CPU cores thanks to Numba parallelism.
PTtools has various utilities built on top of this that simplify the parallel generation of bubbles.
An example of a parallel program is provided below.

\begin{lstlisting}[language=Python]
import numpy as np

from pttools.analysis import BubbleGridVWAlpha
from pttools.bubble import Bubble
from pttools.models import BagModel


def compute(bubble: Bubble):
	if bubble.no_solution_found or bubble.solver_failed:
		return np.nan, np.nan
	return bubble.kappa, bubble.omega

compute.return_type = (float, float)


v_walls = np.linspace(0.1, 0.9, 5)
alpha_ns = np.linspace(0.1, 0.3, 5)
model = BagModel(a_s=1.1, a_b=1, V_s=1)
grid = BubbleGridVWAlpha(model, v_walls, alpha_ns, compute)
bubbles = grid.bubbles
kappas = grid.data[0]
omegas = grid.data[1]
\end{lstlisting}
