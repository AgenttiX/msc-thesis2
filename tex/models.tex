\section{The bag model}
\label{bag_model}
The equation of state is the function $p(T)$ or $p(w)$ that characterises the behaviour of the fluid as a function of temperature $T$ or enthalpy $w$.
It should be highlighted that everything up to this point has been independent of the equation of state,
and we will now move forward to describing our first equation of state.

One of the simplest equations of state is the bag model, where
\cites[eq. 7.33]{lecture_notes}[eq. 8-9]{giese_2020}
\begin{align}
p_s &= a_s T^4 - V_s
\label{eq:bag_ps} \\
p_b &= a_b T^4,
\label{eq:bag_pb}
\end{align}
where $a_s$, $a_b$ and $V_s$ are positive constants with $a_s > a_b$.
The subscripts $s$ refers to the symmetric phase, and $b$ to the broken phase.

In this model the speed of sound simplifies as
\begin{equation}
c_s^2 = \frac{dp}{de} = 1/3,
\end{equation}
This corresponds to assuming both of the phases to be ultrarelativistic.
It also happens to be the same as the $\tilde{v}_-$ limit of (\ref{eq:v_tilde_minus_limit}),
and therefore the Chapman-Jouguet speed $v_{CJ}$ of the bag model is given by (\ref{eq:v_tilde_plus_limit}).

The name of the bag model originates from quantum chromodynamics (QCD), where it's used to describe the proton as a bag of quarks.
\todo{check this}


\section{More complex models}
The bag model includes severe approximations, and for realistic simulations a more complex model is needed.
The primary parameters for these models are
\begin{itemize}
    \item The phase transition parameters: percolation temperature, phase transition duration
    \item Bubble wall velocity
    \item Fraction of energy that is converted into fluid motion: $\kappa$
    \item Numerical prefactor from lattice simulations
\end{itemize}

Speed of sound may not be ultrarelativistic, especially in the broken phase.

\subsection{The constant sound speed model}
The constant sound speed model, also known as the $\mu, \nu$ model, expands the bag model by allowing a change in the speed of sound of the broken phase.
\cites[eq. 15]{giese_2021}[eq. 38]{giese_2020}
\begin{align}
p_s &= \frac{1}{3} a_+ T^\mu - \epsilon \\
p_b &= \frac{1}{3} a_- T^\nu \\
e_s &= a_+ T^\mu + \epsilon \\
e_b &= \frac{1}{3} a_- (\nu - 1) T^\nu
\end{align}
The parameters $a_+$ and $a_-$ are proportional to the number of relativistic degrees of freedom in the symmetric and broken phases, and $\epsilon$ is the temperature-independent vacuum energy that is released in the phase transition.
The parameters $\mu$ and $\nu$ are determined by the sound speeds in the symmetric and broken phases:
\cites[eq. 16]{giese_2021}[eq. 39]{giese_2020}
\begin{align}
\mu = 1 + \frac{1}{c_{s,s}^2}, \\
\nu = 1 + \frac{1}{c_{s,b}^2}.
\end{align}
\footnote{Please note that there is a typo in \cite[eq. 15]{giese_2021}. There the 4 should be a $\mu$.}

\subsection{Model with cubic thermal effects}
Free energy
\todo{Where does this come from? How to extract the thermal effects?}
\cite[eq. 45]{giese_2020}
\begin{align}
\begin{split}
\textsc{F}(\phi, T) =
&- \frac{a_+}{3} T^4 \\
&+ \lambda \left( \phi^4 - 2E\phi^3 T + \phi^2 \left( E^2 T_\text{cr}^2 + c \left( T^2 - T_\text{cr}^2 \right) \right) \right) \\
&+ \frac{\lambda}{4} \left( c - E^2 \right)^2 T_\text{cr}^4
\end{split}
\end{align}

The minimum of the field potential is given by
\cite[eq. 46]{giese_2020}
\begin{equation}
\phi_\text{min} = \frac{3}{4} ET + \sqrt{\frac{T^2}{2}(\frac{9}{8}E^2 - c) - \frac{T_\text{cr}^2}{2} (E^2 - c)}
\end{equation}

\subsection{Model with a two-step phase transition}
Two scalar fields and two symmetry breakings. Examples: electroweak symmetry, $\mathbb{Z}_2$ symmetry.
Cubic term is neglected.
Pressures are of the form
\cite[eq. 47-48]{giese_2020}
\begin{align}
p_s(T) = \frac{1}{3}a_+ T^4 + (b_+ - c_+ T^2)^2 - b_-^2 \\
p_b(T) = \frac{1}{3}a_+ T^4 + (b_- - c_-T^2)^2 - b_-^2
\end{align}
Compare these to the \eqref{eq:bag_ps}, \eqref{eq:bag_pb}.


\section{Energy redistribution}
Efficiency factor $\kappa$, which is the fraction of energy that is converted into fluid motion.
\begin{equation}
\kappa = \frac{3}{\xi_w^3 \Delta \theta} \int d\xi \xi^2 w \gamma^2 v^2
\end{equation}
From solving the hydrodynamic equations of state of a single expanding bubble.

Kinetic energy fraction
\cites[eq. 7.36]{lecture_notes}[eq. 5]{giese_2020}
\begin{align}
K
&= \frac{\rho_\text{fl}}{\bar{e}}
&= \frac{1}{\mathcal{v} \bar{e}} \int d^3 x w \gamma^2 v^2,
\end{align}
where $\mathcal{v}$ is the volume we average over, and $\bar{e}$ is the mean energy density.
It should be noted that $0 < K < 1$.
\cite{giese_2020}

For a single bubble we have, using its volume and the energy density of the symmetric phase that the bubble replaces, the kinetic energy fraction
\cites[eq. 7.37]{lecture_notes}[eq. 5]{giese_2020}
\begin{equation}
K_1 = \frac{3}{\xi_w^3 e_s} \int d\xi \xi^2 w \gamma^2 v^2.
\end{equation}

The kinetic energy fraction can be determined from
\begin{itemize}
\item wall velocity $v_w$
\item the speeds of sound
\item strength parameter of the phase transition
\end{itemize}

Let us define the pseudotrace as
\cites[eq. 34]{giese_2020}[eq. 1]{giese_2021}
\begin{equation}
\overline{\theta} \equiv e - \frac{p}{c_{s,b}^2}.
\end{equation}

For detonations, the optimal phase transition strength is
\cites[eq. 34]{giese_2020}[eq. 1]{giese_2021}
\todo{Todo: Explain how this is defined}
\begin{equation}
\alpha_{\overline{\theta}} \equiv \frac{D \overline{\theta}(T_+)}{3w_+},
\end{equation}


\section{Extensions of the standard model}
Extensions of the Standard model \cite{caprini_detecting_2020}
(mentioned in \cite[p. 14]{lecture_notes})

\section{Noise background}
Non-thermal phenomena: GW production by preheating, when the inflaton decays to SM particles.
\cite{lecture_notes}
