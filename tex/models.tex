\section{General equation of state}
\label{general_eq}
The equation of state is the function $p(T)$ or $p(w)$ that characterises the behaviour of the fluid as a function of temperature $T$ or enthalpy $w$.
It should be highlighted that everything up to this point has been independent of the equation of state.
We will now first describe the derivation for the general equation of state of an ultrarelativistic plasma,
and then introduce approximate models.

\iffalse
In general the pressure of the ultrarelativistic ASDF
\begin{equation}
p(T,\phi) = - \sum_B f_B (m(\phi),T) - \sum_F f_F (m(\phi),T)
\end{equation}
\fi

The energy density at a point is the sum of the energies of all particles at that point.
Therefore the energy density in the position space can be obtained by integrating over the momentum space as
\cite[eq. 4.10]{lecture_notes}
\begin{equation}
e(x) = \int \frac{d^3 p}{(2 \pi)^3} E(p) f(\vec{p}, x).
\end{equation}
\href{https://physics.stackexchange.com/a/141737/}{The factor $(2\pi)^3$ normalises the momentum space}.
Similarly we can obtain the other components of the energy-momentum tensor in the position space as
\cite[eq. 4.13]{lecture_notes}
\begin{equation}
T^{\mu \nu}(x) = \int \frac{d^3 p}{(2 \pi)^3} \frac{p^\mu p^\nu}{p^0} f(\vec{p},x).
\end{equation}
To give an intuitive explanation for this we can note that $p^0 = E = \gamma m_0$ and $p^i = \gamma m_0 v^i$,
Therefore $\frac{p^i}{p^0} = v^i$,
and the factor $\frac{p^i p^j}{p^0} = p^i v^j$ denotes the amount of momentum being transported by the moving plasma.
By inserting this to the definition of pressure \eqref{eq:pressure_from_ep_tensor} we get
\begin{equation}
p(x) = \frac{1}{3} \int \frac{d^3 p}{(2 \pi)^3} \frac{|\vec{p}|^2}{E} f(\vec{p},x)
\end{equation}
This notation is rather confusing, as the $p$ on the left refers to the pressure,
and all $p$ on the right refer to the momentum.

The particle distribution functions for fermions and bosons are
\cite[eq. 4.6]{lecture_notes}
\begin{equation}
f(\vec{p}) = \frac{1}{e^\frac{E-\mu}{T} \pm 1},
\end{equation}
where $+$ gives the Fermi-Dirac distribution for fermions and $-$ gives the Bose-Einstein distribution for bosons.
Let us approximate that the fluid consists entirely of bosons,
and that they are ultrarelativistic: $E(\vec{p}) = \sqrt{\vec{p}^2 + m^2} \approx p$.
Particle number is not conserved in an ultrarelativistic plasma, and therefore $\mu = 0$.
Let us also remember that we can convert the integral to 1D using the area of a sphere $A(r) = 4\pi r^2$.
With these we get
\begin{align}
e &= \frac{1}{2 \pi^2} \int_0^\infty \frac{p^3 dp}{e^\frac{p}{T} - 1},
\label{eq:ultrarelativistic_e} \\
p &= \frac{1}{6 \pi^2} \int_0^\infty \frac{p^3 dp}{e^\frac{p}{T} - 1}.
\label{eq:ultrarelativistic_p}
\end{align}
These contain integrals of the form
\cite[eq. B.36]{schroeder_thermal_2000}
\begin{equation}
\int_0^\infty \frac{x^n}{e^x - 1} = \Gamma(n+1) \zeta(n+1),
\end{equation}
where $\Gamma$ is the
\href{https://en.wikipedia.org/wiki/Gamma_function}{gamma function}, which for integers is $\Gamma(n) = (n-1)!$.
$\zeta$ is the
\href{https://en.wikipedia.org/wiki/Riemann_zeta_function}{Riemann zeta function}.
It can be shown that $\zeta(4) = \frac{\pi^4}{90}$.
\cite[prob. B.19]{schroeder_thermal_2000}
At this ultrarelativistic limit we can see from eq. \eqref{eq:ultrarelativistic_e} and \eqref{eq:ultrarelativistic_p} that $e = 3p$, and therefore using eq. \eqref{eq:cs2_compact} $c_s^2 = \frac{1}{3}$.

To account for the various particle species in the plasma and their non-ultrarelativistic behaviour,
we need to multiply with the degrees of freedom $g_e$ and $g_p$,
which are weighted sums over the contributions of different particle species.
To also account for the field, we need to add its potential $V$ to the final result
\cite[eq. S12]{borsanyi_lattice_2016}
\begin{align}
e(T,\phi) &= \frac{\pi^2}{30} g_e(T) T^4 + V(T,\phi),
\label{eq:e_general} \\
p(T,\phi) &= \frac{\pi^2}{90} g_p(T) T^4 - V(T,\phi).
\label{eq:p_general}
\end{align}

With \eqref{eq:enthalpy_sum}, \eqref{eq:enthalpy_entropy}, \eqref{eq:e_general} and \eqref{eq:p_general} we can obtain the entropy density as
\cite[eq. S12]{borsanyi_lattice_2016}
\begin{equation}
s(T,\phi) = \frac{2\pi^2}{45} g_s(T) T^3,
\end{equation}
\todo{Should the entropy be defined differently? Defining it with $s = \frac{dp}{dT}$ results in a much more complex and less useful expression.}
where the degrees of freedom for the entropy density are
\begin{equation}
g_s = \frac{1}{4} (3g_e + g_p).
\end{equation}
Often only $g_e$ and $g_s$ are provided, and $g_p$ is obtained with
\begin{equation}
g_p = 4g_s - 3g_e.
\end{equation}
For a free scalar particle $g_e = g_p = g_s = 1$, which is known as the Stefan-Boltzmann limit.
\iffalse
due to its reminiscence to the Stefan-Boltzmann law $j^* = \sigma T^4$,
which relates the power radiated by a black body to its temperature with the Stefan-Boltzmann constant $\sigma$.
\fi


\section{The bag model}
\label{bag_model}
For phase transitions in the early universe,
the most commonly used equation of state is the bag model, for which
\begin{equation}
g_{e\pm} = g_{p\pm} = g_{s\pm} = \frac{90}{\pi^2} a_\pm,
\end{equation}
where $a_\pm$ are constants with $a_+ > a_-$.
$+$ refers to the symmetric phase, and $-$ refers to the broken phase.
The potentials $V_\pm$ are also constants with $V_+ > V_-$.
This results in the bag equation of state
\cites[eq. 7.33]{lecture_notes}[eq. 8-9]{giese_2020}
\begin{equation}
p_\pm = a_\pm T^4 - V_\pm,
\label{eq:bag_p}
\end{equation}
The potential difference $\epsilon = V_+ - V_-$ is the temperature-independent vacuum energy that is released in the phase transition.
The potentials are usually shifted so that $V_b = 0$.
\cite{giese_2020}

Using \eqref{eq:entropy_density} we get the entropy density
\begin{equation}
s_\pm = 4 a_\pm T^3,
\end{equation}
and with \eqref{eq:enthalpy_entropy} the enthalpy density
\begin{equation}
w_\pm = 4 a_\pm T^4.
\end{equation}
Finally with \eqref{eq:enthalpy_sum} we get the energy density
\begin{equation}
e_\pm = 3 a_\pm T^4 + V_\pm
\end{equation}

The speed of sound from \eqref{eq:cs2_explicit} simplifies to
\begin{equation}
c_s^2 = \frac{dp}{de} = \frac{dp/dT}{de/dT} = \frac{1}{3}.
\end{equation}
As in section \ref{general_eq}, this corresponds to assuming both of the phases to be ultrarelativistic.
It also happens to be the same as the $\tilde{v}_-$ that corresponds to \eqref{eq:v_tilde_plus_limit}, and therefore the Chapman-Jouguet speed $v_{CJ}$ of the bag model is given by \eqref{eq:v_tilde_plus_limit}.

The trace anomaly of \eqref{eq:theta} simplifies to
\begin{align}
\theta_\pm = V_\pm.
\end{align}
Therefore the transition strength of \eqref{eq:alpha_plus} simplifies to
\begin{equation}
\alpha_+ = \frac{4}{3 w_+} (V_+ - V_-),
\label{eq:alpha_plus_bag}
\end{equation}
and the transition strength at nucleation temperature from \eqref{eq:alpha_n} simplifies to
\begin{equation}
\alpha_n = \frac{4}{3 w_n} (V_+ - V_-).
\label{eq:alpha_n_bag}
\end{equation}
This is trivial to invert, giving enthalpy at the nucleation temperature $w_n$.

The name of the bag model originates from quantum chromodynamics (QCD), where it's used to describe the proton as a bag of quarks.
\todo{check this}


\section{More complex models}
The bag model is a rather crude approximation, and for realistic simulations a more complex model is needed.
In general, the model for an equation of state is constructed by providing two of $g_e, g_p$ and $g_s$,
or equivalently two of $e, p$ and $s$, and the potential $V(T,\phi)$.
The rest of the model properties can be computed from these.

\iffalse
\begin{itemize}
    \item Bubble wall velocity $v_w$
    \item Nucleation temperature $T_n$, or equivalently nucleation enthalpy $w_n$ or transition strength $\alpha_n$
    \item Phase transition duration $\frac{1}{\beta}$ \todo{Check this}
    \item Equation of state
    \item Fraction of energy that is converted into fluid motion: $\kappa$ (???)
    \item Numerical prefactor from lattice simulations (???)
\end{itemize}
\fi

\subsection{The constant sound speed model}
\label{const_cs}
The constant sound speed model, also known as the $\mu_\pm$ model, the $\mu, \nu$ model,
or in some sources as the template model,
expands the bag model by allowing a different speed of sound in each phase.
It is obtained by setting
\begin{equation}
g_{p\pm} = \frac{90}{\pi^2} a_\pm \left( \frac{T}{T_0} \right)^{\mu_\pm - 4},
\end{equation}
where $\mu_\pm$ are constants.
This results in the equation of state
\cites[eq. 15]{giese_2021}[eq. 38]{giese_2020}
\begin{equation}
p_\pm = a_\pm \left( \frac{T}{T_0} \right)^{\mu_\pm - 4} T^4 - V_\pm {\color{gray} \approx a_\pm T^{\mu_\pm} - V_\pm }, \\
\end{equation}
The reference temperature $T_0$ is introduced for the consistency of the units.
For simplicity it can be set to 1 GeV, or whichever unit one is using for the temperature.
The greyed-out version is commonly seen in articles and useful, but has inconsistent units.
The parameters $\mu_\pm$ are determined by the sound speeds in the symmetric and broken phases with
\cites[eq. 16]{giese_2021}[eq. 39]{giese_2020}
\footnote{Please note that there is a typo in \cite[eq. 15]{giese_2021}. There the 4 should be a $\mu$.}
\begin{equation}
\mu_\pm \equiv 1 + \frac{1}{c_{s\pm}^2},
\end{equation}
This model reduces to the bag model,
when $c_{s+}^2 = c_{s-}^2 = \frac{1}{3}$
and equivalently
$\mu_+ = \mu_- = 4$.

Similarly as for the bag model we can derive the entropy density
\begin{equation}
s_\pm = \mu_\pm a_\pm \left( \frac{T}{T_0} \right)^{\mu_\pm - 4} T^3,
\end{equation}
and from it the enthalpy density
\begin{equation}
w_\pm = \mu_\pm a_\pm \left( \frac{T}{T_0} \right)^{\mu_\pm - 4} T^4
\label{eq:w_const_cs}
\end{equation}
and finally the energy density
\begin{equation}
e_\pm = (\mu_\pm - 1) a_\pm \left( \frac{T}{T_0} \right)^{\mu_\pm - 4} T^4 + V_\pm.
\end{equation}
The equation for enthalpy density \eqref{eq:w_const_cs} can be inverted to give the temperature as a function of enthalpy
\begin{equation}
T_\pm = \left( \frac{w}{\mu_\pm a_\pm T_0^4} \right)^\frac{1}{\mu_\pm} T_0,
\end{equation}
Therefore the effective degrees of freedom are
\begin{align}
g_{p\pm} &= \frac{90}{\pi^2} a_\pm \left( \frac{T}{T_0} \right)^{\mu_\pm - 4}, \\
g_{s\pm} &= \frac{45}{2\pi^2} \mu_\pm a_\pm \left( \frac{T}{T_0} \right)^{\mu_\pm - 4}, \\
g_{e\pm} &= \frac{30}{\pi^2} (\mu - 1) a_\pm \left( \frac{T}{T_0} \right)^{\mu_\pm - 4}.
\end{align}


The trace anomaly of \eqref{eq:theta} simplifies to
\begin{align}
\theta_\pm(w)
&= \left( \frac{\mu_\pm}{4} - 1 \right) a_\pm \left( \frac{T}{T_0} \right)^{\mu_\pm - 4} T^4 + V_\pm \\
&= \left( \frac{1}{4} - \frac{1}{\mu_\pm} \right) w_\pm + V_\pm
\label{eq:theta_const_cs}
\end{align}
The transition strength of \eqref{eq:alpha_plus} simplifies to
\begin{equation}
\alpha_+ = \frac{1}{3} \left( 1 - \frac{4}{\mu_+} \right) - \frac{1}{3} \left(1 - \frac{4}{\mu_-} \right) \frac{w_-}{w_+} + \alpha_{+,\text{bag}}.
\label{eq:alpha_plus_const_cs}
\end{equation}
The transition strength at nucleation temperature of \eqref{eq:alpha_n} is correspondingly
\begin{align}
\alpha_n &= \frac{1}{3} \left( 1 - \frac{4}{\mu_+} \right) - \frac{1}{3} \left(1 - \frac{4}{\mu_-} \right) \frac{w(T_n, \phi_-)}{w_n} + \alpha_{n,\text{bag}} \\
&= \frac{4}{3} \left( \frac{1}{\mu_-} - \frac{1}{\mu_+} \right) + \frac{1}{3} \left( 1 - \frac{4}{\mu_-} \right)
\left( 1 - \frac{\mu_- a_-}{\mu_+ a_+} \left( \frac{T}{T_0} \right)^{\mu_- - \mu_+} \right) + \alpha_{n,\text{bag}}
\label{eq:alpha_n_const_cs}
\end{align}
\iffalse
% This is wrong
Inverting this gives the enthalpy at the nucleation temperature
\begin{equation}
w_n = \frac{V_+ - V_-}{ \frac{3}{4} \alpha_n + \frac{1}{\mu_+} - \frac{1}{\mu_-} }.
\label{eq:wn_const_cs}
\end{equation}
\fi
When $\mu_- = 4$, $\alpha_n$ of \eqref{eq:alpha_n_const_cs} becomes independent of $w_-$, resulting in
\begin{equation}
w_n = \frac{4 (V_+ - V_-)}{3 \alpha_n + \frac{4}{\mu_+} - 1}.
\label{eq:wn_const_cs_mu4}
\end{equation}
In this case we can obtain $v_{CJ}(\alpha_n, c_{s+})$ analytically.
First we insert $\alpha_n$ to \eqref{eq:wn_const_cs_mu4} to get $w_n$,
and then insert $w_n$ to \eqref{eq:alpha_plus_const_cs} to get $\alpha_+$,
since in this case $\alpha_+$ is independent of $w_-$.
Then we can insert $c_{s-}$ and $\alpha_+$ to \eqref{eq:v_tilde_plus} to get $\tilde{v}_+ = v_{CJ}$.

\iffalse
Now we have everything necessary to compute $\alpha_+$ and the Chapman-Jouguet speed $v_{CJ}$ analytically from $c_{s+}$, $c_{s-}$ and $\alpha_n$.
From these we get $w_n$ with \eqref{eq:wn_const_cs}, and for a detonation $w_+=w_n$.
This we can insert to \eqref{eq:wm_const_cs}, giving us $w_-$.
Now we have all the variables necessary to compute $\alpha_+$ with \eqref{eq:alpha_plus}.
Then we insert $\tilde{v}_- = c_{s-}$ to \eqref{eq:v_tilde_plus}, giving us the Chapman-Jouguet speed $v_{CJ} = v_+$.
\todo{This is wrong! We have a circular dependency and must solve the equation group at once.}
\fi

The pseudotrace of \eqref{eq:theta_bar} is given by
\begin{align}
\bar{\theta} = (\mu_\pm - \mu_-) a \left(\frac{T}{T_0}\right)^{\mu_\pm - 4} T_0^4 + \mu_- V_\pm.
\end{align}
Therefore the corresponding transition strength of \eqref{eq:alpha_theta_bar_plus} becomes
\begin{equation}
\alpha_{\bar{\theta}_+} = \frac{1}{3}\left(1 - \frac{\mu_-}{\mu_+}\right) + \frac{\mu_-}{3w_+} \Delta V.
\end{equation}

If we set $T_0 = T_c$, eq. \eqref{eq:critical_temp} results in
\begin{equation}
\Delta V = (a_+ - a_-) T_c^4.
\end{equation}
Using this the "bag" part in eq. \ref{eq:alpha_n_const_cs} becomes
\begin{equation}
\alpha_{n,\text{bag}} = \frac{4}{3 \mu_+} \left( 1 - \frac{a_-}{a_+} \right) \left( \frac{T}{T_c} \right)^{-\mu_+}.
\end{equation}
Using this we can reorder \eqref{eq:alpha_n_const_cs} to
\begin{equation}
\frac{a_-}{a_+} = \frac{4 + \left( \mu_+ - 4 - 3 \mu_+ \alpha_n \right) \left(\frac{T_n}{T_c}\right)^{\mu_+}}{4 + \left( \mu_- - 4 \right) \left(\frac{T_n}{T_c}\right)^{\mu_-}}.
\end{equation}
Noting that $T_n < T_c$, this restricts $\alpha_n$ to
\begin{equation}
\alpha_n > \frac{\mu_+ - \mu_-}{3 \mu_+}.
\end{equation}
Even though we have used $T_0 = T_c$ in the intermediate steps, this restriction is general.

With $T_0 = T_c$ we also have
\begin{equation}
\alpha_{\bar{\theta}_+ = \frac{1}{3} \left(1 - \frac{\mu_-}{\mu_+}\right) + \frac{1}{3} \frac{\mu_-}{\mu_+} \left(1 - \frac{a_-}{a_+} \right) \frac{T}{T_c}\right)^{-\mu_+}
\end{equation}



\subsection{Model with cubic thermal effects}
Free energy
\todo{Where does this come from? How to extract the thermal effects?}
\cite[eq. 45]{giese_2020}
\begin{align}
\begin{split}
\textsc{F}(\phi, T) =
&- \frac{a_+}{3} T^4 \\
&+ \lambda \left( \phi^4 - 2E\phi^3 T + \phi^2 \left( E^2 T_\text{cr}^2 + c \left( T^2 - T_\text{cr}^2 \right) \right) \right) \\
&+ \frac{\lambda}{4} \left( c - E^2 \right)^2 T_\text{cr}^4
\end{split}
\end{align}

The minimum of the field potential is given by
\cite[eq. 46]{giese_2020}
\begin{equation}
\phi_\text{min} = \frac{3}{4} ET + \sqrt{\frac{T^2}{2}(\frac{9}{8}E^2 - c) - \frac{T_\text{cr}^2}{2} (E^2 - c)}
\end{equation}


\subsection{Model with a two-step phase transition}
Two scalar fields and two symmetry breakings. Examples: electroweak symmetry, $\mathbb{Z}_2$ symmetry.
Cubic term is neglected.
Pressures are of the form
\cite[eq. 47-48]{giese_2020}
\begin{align}
p_s(T) = \frac{1}{3}a_+ T^4 + (b_+ - c_+ T^2)^2 - b_-^2 \\
p_b(T) = \frac{1}{3}a_+ T^4 + (b_- - c_-T^2)^2 - b_-^2
\end{align}
Compare these to the \eqref{eq:bag_ps}, \eqref{eq:bag_pb}.


\subsection{Standard Model}
In the Standard Model the electroweak phase transition is not of first order but a crossover,
and therefore there is no bubble nucleation.
However, it serves as an important baseline.
The $g_e$ and $g_s$ data for the Standard Model is given by \cite{borsanyi_lattice_2016}.


\section{Energy redistribution}
\label{energy_redistribution}
Once the velocity profile of a bubble is known, it can be used to compute various thermodynamic quantities.

We have assumed that the background spacetime is Minkowski, and that the spacetime does not expand significantly during the phase transition.
Therefore, the total energy of the system is conserved, and the average energy density $\bar{e}$ over the bubble is the same as outside it,
\begin{equation}
\bar{e} = 4 \pi \int_0^R dr r^2 T^{00} = e(w_n, \phi_s),
\label{eq:e_conservation}
\end{equation}
for $R$ larger than the fluid velocity profile.
\todo{Energy-momentum of the fluid is not conserved, but for the total system it is. See lecture notes p. 21.}

For an ideal fluid of eq. \eqref{eq:ep_tensor}
% This is easy to prove by hand.
\begin{equation}
T^{00} = w\gamma^2 - p = w\gamma^2 v^2 + e = w\gamma^2 v^2 + \frac{3}{4}w + \theta.
\end{equation}

Inserting this to \eqref{eq:e_conservation} results in the equation
\begin{equation}
e_K + \Delta e_Q = - \Delta e_\theta,
\end{equation}
which consists of the terms defined below.
The kinetic energy density is given by
\begin{equation}
e_K \equiv 4 \pi \int_0^{\xi_\text{max}} d\xi \xi^2 w \gamma^2 v^2.
\end{equation}
It denotes the energy that is converted to macroscopic fluid movement.
The thermal energy density difference is given by
\begin{equation}
\Delta e_Q \equiv 4 \pi \int_0^{\xi_\text{max}} d\xi \xi^2 \frac{3}{4} (w - w_n).
\end{equation}
It denotes the energy that is converted to microscopic fluid movement.
The trace anomaly difference is given by
\begin{equation}
\Delta e_\theta \equiv 4 \pi \int_0^{\xi_\text{max}} d\xi \xi^2 (\theta - \theta_n).
\end{equation}
It can be considered as the potential energy available for transformation.
The upper integration limit $\xi_\text{max}$ can be chosen arbitrarily as long as it's outside the bubble, as outside the bubble all three quantities go to zero outside the fluid shell due to $v=0$, $w=w_n$ and $\theta = \theta_n$.
A convenient choice is $\xi_\text{max} = \max (v_w, \xi_{sh})$.

It should be noted that the trace anomaly is not quite equivalent to the thermal potential energy density $V_T(T,\phi)$, as using eq. \eqref{eq:enthalpy_pressure}, \eqref{eq:enthalpy_sum} and \eqref{eq:p_general} we can see that for the case of temperature-independent $g$,
\begin{equation}
\theta = V_T - \frac{1}{4} T \frac{\partial V_T}{\partial T}.
\end{equation}
Therefore, not all of the potential energy difference can be turned into kinetic and thermal energy.
\cite[ch. B.2]{hindmarsh_gw_pt_2019}

The fraction of total energy that is converted to kinetic energy is known as the kinetic energy fraction,
\begin{equation}
K \equiv \frac{e_K}{\bar{e}}.
\end{equation}

Kinetic and thermal efficiency factors quantify the fraction of the available energy $\Delta e_\theta$ that is converted to kinetic and thermal energy.
They can be defined as
\begin{equation}
\kappa \equiv \frac{e_K}{| \Delta e_\theta |}, \quad
\omega \equiv \frac{\Delta e_Q}{\Delta e_\theta}.
\end{equation}
They have the property that $\kappa + \omega = 1$.

We can obtain the volume-average entropy density similarly as the energy density, with
\begin{equation}
s_\text{avg} = \frac{1}{v_w^3} \int \left( s(w,\phi) - s(w_n, \phi_s) \right) d\xi^3
\end{equation}
\todo{Should there be a delta here? And change the integral to 1D.}

Relative volume-averaged entropy density
\begin{equation}
TODO = \frac{s_\text{avg}}{s_n}
\end{equation}

Enthalpy-weighted mean square fluid 4-velocity around the bubble
\begin{equation}
\bar{U}_f^2 = \frac{\Delta e_Q}{w_n}.
\end{equation}
\todo{Add a generic definition}
The \href{https://en.wikipedia.org/wiki/Heat_capacity_ratio}{adiabatic index} is defined as
\begin{equation}
\gamma \equiv \frac{c_P}{c_V},
\end{equation}
where $c_P$ is the specific heat capacity at constant pressure,
and $c_V$ is the specific heat capacity at constant volume.
For a classical monoatomic fluid $\gamma = \frac{5}{3}$,
and for an ultrarelativistic fluid $\gamma = \frac{4}{3}$.

The enthalpy-weighted mean square fluid 4-velocity around the bubble and the kinetic energy density fraction are linked by
\begin{equation}
K = \Gamma \bar{U}_f^2,
\end{equation}
where
\begin{equation}
\Gamma \equiv \frac{\bar{w}}{\bar{e}}
\end{equation}
In the case of the ultrarelativistic bag model $\Gamma$ is the mean adiabatic index,
but for a more general model this may not be the case.
\todo{Define wbar}


\iffalse
Efficiency factor $\kappa$, which is the fraction of energy that is converted into fluid motion.
\begin{equation}
\kappa \equiv \frac{3}{\xi_w^3 \Delta \theta} \int d\xi \xi^2 w \gamma^2 v^2
\end{equation}
From solving the hydrodynamic equations of state of a single expanding bubble.

Kinetic energy fraction
\cites[eq. 7.36]{lecture_notes}[eq. 5]{giese_2020}
\begin{align}
K
&\equiv \frac{\rho_\text{fl}}{\bar{e}}
= \frac{1}{\mathcal{v} \bar{e}} \int d^3 x w \gamma^2 v^2,
\end{align}
where $\mathcal{v}$ is the volume we average over, and $\bar{e}$ is the mean energy density.
It should be noted that $0 < K < 1$.
\cite{giese_2020}

For a single bubble we have, using its volume and the energy density of the symmetric phase that the bubble replaces, the kinetic energy fraction
\cites[eq. 7.37]{lecture_notes}[eq. 5]{giese_2020}
\begin{equation}
K_1 = \frac{3}{\xi_w^3 e_s} \int d\xi \xi^2 w \gamma^2 v^2.
\end{equation}

The kinetic energy fraction can be determined from
\begin{itemize}
\item wall velocity $v_w$
\item the speeds of sound
\item strength parameter of the phase transition
\end{itemize}
\fi


\section{Extensions of the standard model}
Extensions of the Standard model \cite{caprini_detecting_2020}
(mentioned in \cite[p. 14]{lecture_notes})

\section{Noise background}
Non-thermal phenomena: GW production by preheating, when the inflaton decays to SM particles.
\cite{lecture_notes}
