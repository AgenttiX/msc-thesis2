\section{General model}
\iffalse
In general the pressure of the ultrarelativistic ASDF
\begin{equation}
p(T,\phi) = - \sum_B f_B (m(\phi),T) - \sum_F f_F (m(\phi),T)
\end{equation}
\fi

Energy density can be obtained by integrating over the momentum space as
\begin{equation}
e = g \int \frac{d^3 p}{(2 \pi)^3} E(\vec{p}) f(\vec{p}).
\end{equation}
The particle distribution functions of statistical physics is
\begin{equation}
f(\vec{p}) = \frac{1}{e^{E-\mu}{T} \pm 1},
\end{equation}
where $+$ is for fermions and $-$ is for bosons.
Let us assume that the fluid consists of bosons,
and that they are ultrarelativistic: $E(\vec{p}) \approx p$, $\mu \approx 0$ and $E = \sqrt{p^2 + m^2} \approx p$.
Let us also remember that we can convert the integral to 1D using the area of a sphere $A(r) = 4\pi r^2$.
With these approximations we get
\begin{equation}
e = \frac{g}{2 \pi^2} \int_0^\infty \frac{p^3 dp}{e^\frac{p}{T} - 1}.
\end{equation}
This contains an integral of the form
\cite[eq. B.36]{schroeder_thermal_2000}
\begin{equation}
\int_0^\infty \frac{x^n}{e^x - 1} = \Gamma(n+1) \zeta(n+1).
\end{equation}
It can be proven that $\zeta(4) = \frac{\pi^2}{90}$.
\cite[prob. B.19]{schroeder_thermal_2000}
We also need to add the potential $V$ of the field, resulting in
\cite[eq. S12]{borsanyi_lattice_2016}
\begin{equation}
e(T,\phi) = \frac{\pi^2}{30} g_e(T,\phi) T^4.
\end{equation}
This is also known as the Stefan-Boltzmann limit of a single scalar field due to its reminiscence to the Stefan-Boltzmann law $j^* = \sigma T^4$,
which relates the power radiated by a black body to its temperature with the Stefan-Boltzmann constant $\sigma$.

The pressure is defined as
\begin{equation}
p = g \int \frac{d^3 p}{(2 \pi)^3} \frac{f(\vec{p}) |\vec{p}|^2}{3E}.
\end{equation}
With similar steps we get
\begin{equation}
p(T,\phi) = \frac{\pi^2}{90} g_p(T,\phi) T^4 - V(T,\phi).
\end{equation}

Entropy density
\cite[eq. S12]{borsanyi_lattice_2016}
\begin{equation}
s(T,\phi) = \frac{2\pi^2}{45} g_s(T,\phi) T^3 = 4 \frac{\pi^2}{90} g_s(T,\phi)
\end{equation}

The equations of section \ref{rel_hydro} hold, when $g_e = g_p = g_s$.
However, we can account for non-idealities by defining them separately.
\todo{Check this}


\section{The bag model}
\label{bag_model}
The equation of state is the function $p(T)$ or $p(w)$ that characterises the behaviour of the fluid as a function of temperature $T$ or enthalpy $w$.
It should be highlighted that everything up to this point has been independent of the equation of state,
and we will now move forward to describing our first equation of state.

One of the simplest equations of state is the bag model, where
\todo{Check from Giese whether this is equivalent to "radiation fluid"}
\cites[eq. 7.33]{lecture_notes}[eq. 8-9]{giese_2020}
\begin{align}
p_s &= a_s T^4 - V_s,
\label{eq:bag_ps} \\
p_b &= a_b T^4 - V_b,
\label{eq:bag_pb}
\end{align}
where $a_s$, $a_b$, $V_s$ and $V_b$ are positive constants with $a_s > a_b$ and $V_s > V_b$.
The subscripts $s$ refers to the symmetric phase, and $b$ to the broken phase.
The potential difference $V_s - V_b$ is the temperature-independent vacuum energy that is released in the phase transition.
In some articles it's denoted with $\epsilon$.
\todo{Add citation to Giese?}
The potentials are usually shifted so that $V_b = 0$.
The constants $a_s$ and $a_b$ are simplified from the relativistic degrees of freedom as
\begin{align}
a_s = \frac{\pi^2}{90} g_s, \\
a_b = \frac{\pi^2}{90} g_b.
\end{align}

Using \eqref{eq:energy_density} we get the energy density
\begin{align}
e_s &= 4 a_s T^4 + V_s, \\
e_b &= 4 a_b T^4 + V_b,
\end{align}
and with \eqref{eq:enthalpy_sum} the enthalpy density
\begin{align}
w_s &= 4 a_s T^4, \\
w_b &= 4 a_b T^4.
\end{align}

The speed of sound from \eqref{eq:cs2_explicit} simplifies to
\begin{equation}
c_s^2 = \frac{dp}{de} = \frac{dp/dT}{de/dT} = \frac{1}{3}.
\end{equation}
This corresponds to assuming both of the phases to be ultrarelativistic.
It also happens to be the same as the $\tilde{v}_-$ that corresponds to \eqref{eq:v_tilde_plus_limit}, and therefore the Chapman-Jouguet speed $v_{CJ}$ of the bag model is given by \eqref{eq:v_tilde_plus_limit}.

The trace anomaly of \eqref{eq:theta} simplifies to
\begin{align}
\theta_s = V_s, \\
\theta_b = V_b.
\end{align}
Therefore the transition strengths of \eqref{eq:alpha_plus} and \eqref{eq:alpha_n} simplify to
\begin{equation}
\alpha_+ = \alpha_n = \frac{4}{3 w_n} (V_s - V_b).
\label{eq:alpha_n_bag}
\end{equation}
This is trivial to invert, giving enthalpy at the nucleation temperature $w_n$.

The name of the bag model originates from quantum chromodynamics (QCD), where it's used to describe the proton as a bag of quarks.
\todo{check this}


\section{More complex models}
The bag model is a rather crude approximation, and for realistic simulations a more complex model is needed.
The primary parameters for these models are
\begin{itemize}
    \item The phase transition parameters: percolation temperature, phase transition duration
    \item Bubble wall velocity $v_w$
    \item Fraction of energy that is converted into fluid motion: $\kappa$
    \item Numerical prefactor from lattice simulations
\end{itemize}

Speed of sound may not be ultrarelativistic, especially in the broken phase.

\subsection{The constant sound speed model}
The constant sound speed model, also known as the $\mu, \nu$ model,
expands the bag model by allowing a different speed of sound in each phase.
\cites[eq. 15]{giese_2021}[eq. 38]{giese_2020}
\begin{align}
p_s &= a_s \left( \frac{T}{T_0} \right)^{\mu-4} T^4 - V_s {\color{gray} \approx a_s T^\mu - V_s }, \\
p_b &= a_b \left( \frac{T}{T_0} \right)^{\nu-4} T^4 - V_b {\color{gray} \approx a_b T^\nu - V_b}.
\end{align}
The reference temperature $T_0$ is introduce for the consistency of the units.
For simplicity it can be set to 1 GeV, or whichever unit one is using for the temperature.
The greyed-out versions are commonly seen in articles and useful, but have inconsistent units.
The parameters $\mu$ and $\nu$ are determined by the sound speeds in the symmetric and broken phases:
\cites[eq. 16]{giese_2021}[eq. 39]{giese_2020}
\footnote{Please note that there is a typo in \cite[eq. 15]{giese_2021}. There the 4 should be a $\mu$.}
\begin{align}
\mu = 1 + \frac{1}{c_{ss}^2}, \\
\nu = 1 + \frac{1}{c_{sb}^2}.
\end{align}
This model reduces to the bag model,
when $c_{ss}^2 = c_{sb}^2 = \frac{1}{3}$.

Similarly as for the bag model we can derive the energy density
\begin{align}
e_s &= (\mu - 1) a_s \left( \frac{T}{T_0} \right)^{\mu-4} T^4 + V_s, \\
e_b &= (\nu - 1) a_b \left( \frac{T}{T_0} \right)^{\mu-4} T^4 + V_b.
\end{align}
The enthalpy density is given by
\begin{align}
w_s &= \mu a_s \left( \frac{T}{T_0} \right)^{\mu-4} T^4, \\
w_b &= \nu a_b \left( \frac{T}{T_0} \right)^{\nu-4} T^4.
\end{align}
This can be inverted to give the temperature as a function of enthalpy
\begin{align}
T_s &= T_0 \left( \frac{w}{\mu a_s T_0^4} \right)^\frac{1}{\mu}, \\
T_b &= T_0 \left( \frac{w}{\nu a_s T_0^4} \right)^\frac{1}{\nu},
\end{align}
and the entropy density
\begin{align}
s_s &= \mu a_s \left( \frac{T}{T_0} \right)^{\mu-4} T_0^3, \\
s_b &= \nu a_b \left( \frac{T}{T_0} \right)^{\mu-4} T_0^3.
\end{align}
The trace anomaly of \eqref{eq:theta} simplifies to
\begin{align}
\theta_s(w) = \left( \frac{1}{4} - \frac{1}{\mu} \right) w + V_s, \\
\theta_b(w) = \left( \frac{1}{4} - \frac{1}{\nu} \right) w + V_b
\end{align}
With this the transition strength at nucleation temperature of \eqref{eq:alpha_n} simplifies to
\begin{equation}
\alpha_n = \frac{4}{3} \left( \frac{1}{\nu} - \frac{1}{\mu} + \frac{1}{w_n} (V_s - V_b) \right)
\end{equation}
Inverting this gives the enthalpy at the nucleation temperature
\begin{equation}
w_n = \frac{V_s - V_b}{ \frac{3}{4} \alpha_n + \frac{1}{\mu} - \frac{1}{\nu} }.
\label{eq:wn_const_cs}
\end{equation}
We can rearrange the first junction condition \eqref{eq:junction_condition_1} to
\begin{equation}
w_- = w_+ \frac{\tilde{\gamma}_+^2 \tilde{v}_+}{\tilde{\gamma}_-^2 \tilde{v}_-}.
\label{eq:wm_const_cs}
\end{equation}
\todo{Check that this is correct in PTtools}

Now we have everything necessary to compute $\alpha_+$ and the Chapman-Jouguet speed $v_{CJ}$ analytically from $c_{ss}$, $c_{sb}$ and $\alpha_n$.
From these we get $w_n$ with \eqref{eq:wn_const_cs}, and for a detonation $w_+=w_n$.
This we can insert to \eqref{eq:wm_const_cs}, giving us $w_-$.
Now we have all the variables necessary to compute $\alpha_+$ with \eqref{eq:alpha_plus}.
Then we insert $\tilde{v}_- = c_{sb}$ to \eqref{eq:v_tilde_plus}, giving us the Chapman-Jouguet speed $v_{CJ} = v_+$.
\todo{This is wrong! We have a circular dependency and must solve the equation group at once.}


\subsection{Model with cubic thermal effects}
Free energy
\todo{Where does this come from? How to extract the thermal effects?}
\cite[eq. 45]{giese_2020}
\begin{align}
\begin{split}
\textsc{F}(\phi, T) =
&- \frac{a_+}{3} T^4 \\
&+ \lambda \left( \phi^4 - 2E\phi^3 T + \phi^2 \left( E^2 T_\text{cr}^2 + c \left( T^2 - T_\text{cr}^2 \right) \right) \right) \\
&+ \frac{\lambda}{4} \left( c - E^2 \right)^2 T_\text{cr}^4
\end{split}
\end{align}

The minimum of the field potential is given by
\cite[eq. 46]{giese_2020}
\begin{equation}
\phi_\text{min} = \frac{3}{4} ET + \sqrt{\frac{T^2}{2}(\frac{9}{8}E^2 - c) - \frac{T_\text{cr}^2}{2} (E^2 - c)}
\end{equation}


\subsection{Model with a two-step phase transition}
Two scalar fields and two symmetry breakings. Examples: electroweak symmetry, $\mathbb{Z}_2$ symmetry.
Cubic term is neglected.
Pressures are of the form
\cite[eq. 47-48]{giese_2020}
\begin{align}
p_s(T) = \frac{1}{3}a_+ T^4 + (b_+ - c_+ T^2)^2 - b_-^2 \\
p_b(T) = \frac{1}{3}a_+ T^4 + (b_- - c_-T^2)^2 - b_-^2
\end{align}
Compare these to the \eqref{eq:bag_ps}, \eqref{eq:bag_pb}.


\section{Energy redistribution}
Efficiency factor $\kappa$, which is the fraction of energy that is converted into fluid motion.
\begin{equation}
\kappa = \frac{3}{\xi_w^3 \Delta \theta} \int d\xi \xi^2 w \gamma^2 v^2
\end{equation}
From solving the hydrodynamic equations of state of a single expanding bubble.

Kinetic energy fraction
\cites[eq. 7.36]{lecture_notes}[eq. 5]{giese_2020}
\begin{align}
K
&= \frac{\rho_\text{fl}}{\bar{e}}
&= \frac{1}{\mathcal{v} \bar{e}} \int d^3 x w \gamma^2 v^2,
\end{align}
where $\mathcal{v}$ is the volume we average over, and $\bar{e}$ is the mean energy density.
It should be noted that $0 < K < 1$.
\cite{giese_2020}

For a single bubble we have, using its volume and the energy density of the symmetric phase that the bubble replaces, the kinetic energy fraction
\cites[eq. 7.37]{lecture_notes}[eq. 5]{giese_2020}
\begin{equation}
K_1 = \frac{3}{\xi_w^3 e_s} \int d\xi \xi^2 w \gamma^2 v^2.
\end{equation}

The kinetic energy fraction can be determined from
\begin{itemize}
\item wall velocity $v_w$
\item the speeds of sound
\item strength parameter of the phase transition
\end{itemize}

Let us define the pseudotrace as
\cites[eq. 34]{giese_2020}[eq. 1]{giese_2021}
\begin{equation}
\overline{\theta} \equiv e - \frac{p}{c_{s,b}^2}.
\end{equation}

For detonations, the optimal phase transition strength is
\cites[eq. 34]{giese_2020}[eq. 1]{giese_2021}
\todo{Todo: Explain how this is defined}
\begin{equation}
\alpha_{\overline{\theta}} \equiv \frac{D \overline{\theta}(T_+)}{3w_+},
\end{equation}


\section{Extensions of the standard model}
Extensions of the Standard model \cite{caprini_detecting_2020}
(mentioned in \cite[p. 14]{lecture_notes})

\section{Noise background}
Non-thermal phenomena: GW production by preheating, when the inflaton decays to SM particles.
\cite{lecture_notes}
