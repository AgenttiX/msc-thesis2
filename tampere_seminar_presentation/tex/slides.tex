\begin{frame}
    \titlepage
\end{frame}

\begin{frame}
    \frametitle{Outline}
    \tableofcontents
\end{frame}

\section{Introduction}

\begin{frame}
    \frametitle{First-order phase transitions}
    \begin{minipage}[t]{0.48\linewidth}
        \begin{itemize}
            \item Spontaneous tunneling to the new phase \textrightarrow \ bubble nucleation and expansion
            \item Bubble expansion is governed by relativistic hydrodynamics
            \item Energy-momentum conservation gives rise to
            \begin{itemize}
                \item Wave equation
                \item Bubble wall junction conditions
                \item Continuity equations, aka. hydrodynamic equations
            \end{itemize}
        \end{itemize}
    \end{minipage}%
    \hfill%
    \begin{minipage}[t]{0.48\linewidth}
        \includegraphics[width=0.25\textwidth]{../fig/HiggsBubble1.pdf}%
        \includegraphics[width=0.25\textwidth]{../fig/HiggsBubble2.pdf}%
        \includegraphics[width=0.25\textwidth]{../fig/HiggsBubble3.pdf} \\
        TODO fix this positioning
    \end{minipage}
\end{frame}

\section{Phase transition bubble hydrodynamics}

\begin{frame}
    \frametitle{Dimensionality of the problem}
    \begin{itemize}
        \item Self-similarity
        \begin{itemize}
            \item Friction results in a constant wall speed $v_\text{wall}$
            \item As the bubble expands, its shape stays the same
            \item \textrightarrow time-independent solution
        \end{itemize}
        \item Spherical symmetry
        \item 3+1 dimensional problem reduces to time-independent 1D
    \end{itemize}
\end{frame}

\begin{frame}
    \frametitle{Energy-momentum tensor}
    \begin{itemize}
        \item In Minkowski space and Cartesian coordinates
        \begin{equation}
        T_{\mu \nu} =
        \begin{bmatrix}
        e & -q_1 & -q_2 & -q_3 \\
        -q_1 & p + \Pi_{11} & \Pi_{12} & \Pi_{13} \\
        -q_2 & \Pi_{12} & p + \Pi_{22} & \Pi_{23} \\
        -q_3 & \Pi_{13} & \Pi_{23} & p + \Pi_{33}
        \end{bmatrix},
        \label{eq:ep_tensor_general_matrix}
        \end{equation}
        \item Ideal fluid: in the rest frame of the fluid $T^{0j} = 0, \quad i \neq j: \ T^{ij} = 0$
        \begin{equation}
        T_{\mu \nu} =
        \begin{bmatrix}
        e & 0 & 0 & 0 \\
        0 & p & 0 & 0 \\
        0 & 0 & p & 0 \\
        0 & 0 & 0 & p
        \end{bmatrix}
        \quad \quad
        T^{\mu \nu}_f = (e+p) u^\mu u^\nu + p g^{\mu \nu}
        \end{equation}
        \item Pressure $p = \frac{1}{3} T^i_i$
    \end{itemize}
\end{frame}

\begin{frame}
    \frametitle{Wave equation}
    \begin{itemize}
        \item Constant background space-time \textrightarrow \ energy-momentum conservation $\nabla_\mu T^{\mu\nu} = 0$
        \item Energy-momentum tensor of an ideal fluid
        \begin{equation}
            T^{\mu \nu}_f = (e+p) u^\mu u^\nu + p g^{\mu \nu}
        \end{equation}
        \item For a one-dimensional flow in Cartesian coordinates
        \begin{align}
            \partial_t \left[ (e+pv^2) \gamma^2 \right] + \partial_x \left[ (e+p) \gamma^2 v \right] &= 0,
            \label{eq:ep_conservation_1d_1} \\
            \partial_t \left[ (e+p) \gamma^2 v \right] + \partial_x \left[ (ev^2 + p) \gamma^2 \right] &= 0
            \label{eq:ep_conservation_1d_2}
        \end{align}
        \item First-order perturbation \textrightarrow \ wave equation with speed of sound
        \begin{equation}
            \partial_t^2 (\delta e) - \frac{\delta p}{\delta e} \partial_x^2(\delta e) = 0
            \quad \quad
            c_s^2 \equiv \frac{dp}{de} = \frac{dp/dT}{de/dT}
        \end{equation}
    \end{itemize}
\end{frame}

\begin{frame}
    \frametitle{Phase boundary}
    \begin{itemize}
        \item Energy-momentum conservation
        \begin{equation}
            \nabla_\mu T^{\mu\nu} = 0
            \quad \Rightarrow \quad
            \partial_z T^{zz} = \partial_z T^{z0} = 0
        \end{equation}
        \item Inserting ideal fluid $T^{\mu \nu}_f = (e+p) u^\mu u^\nu + p g^{\mu \nu}$ \textrightarrow \ Junction conditions
        \begin{align}
            w_- \tilde{\gamma}_-^2 \tilde{v}_- &= w_+ \tilde{\gamma}_+^2 \tilde{v}_+
            \label{eq:junction_condition_1} \\
            w_- \tilde{\gamma}_-^2 \tilde{v}_-^2 + p_- &= w_+ \tilde{\gamma}_+^2 \tilde{v}_+^2 + p_+
            \label{eq:junction_condition_2}
        \end{align}
        \item By defining new variables $\theta = \frac{1}{4}(e-3p), \quad \textcolor{red}{\alpha_+} \equiv \frac{4}{3} \frac{\theta_+(w_+) - \theta_-(w_-)}{w_+}, \quad r = \frac{w_+}{w_-}$
        \begin{align}
            \tilde{v}_+ &= \frac{1}{2(1+\textcolor{red}{\alpha_+})}\left[ \left(\frac{1}{3\tilde{v}_-}+\tilde{v}_-\right) \pm \sqrt{\left(\frac{1}{3\tilde{v}_-} - \tilde{v}_- \right)^2 + 4\textcolor{red}{\alpha_+}^2 + \frac{8}{3} \textcolor{red}{\alpha_+}} \right],
            \label{eq:v_tilde_plus}
            \\
            \tilde{v}_- &= \frac{1}{2} \left[ \left( (1+\textcolor{red}{\alpha_+})\tilde{v}_+ + \frac{1-3\alpha_+}{3\tilde{v}_+} \right) \pm \sqrt{\left((1+\textcolor{red}{\alpha_+})\tilde{v}_+ + \frac{1-3\textcolor{red}{\alpha_+}}{3\tilde{v}_+} \right)^2 - \frac{4}{3}} \right).
            \label{eq:v_tilde_minus}
        \end{align}
    \end{itemize}
\end{frame}

\begin{frame}
    \frametitle{Hydrodynamic equations}
    \begin{itemize}
        \item Energy-momentum conservation $\nabla_\mu T^{\mu\nu} = 0$
        \item Projection \textrightarrow \ hydrodynamic equations
        \begin{align}
            0 &= u_\mu \partial_\nu T^{\mu \nu} = -\partial_\mu (w u^\mu) + u^\mu \partial_\mu p, \\
            0 &= \bar{u}_\mu \partial_\nu T^{\mu \nu} = w \bar{u}^\nu u^\mu \partial_\mu u_\nu + \bar{u}^\mu \partial_\mu p.
        \end{align}
        \item Using self-similarity $\xi = \frac{r}{t}$
        \begin{align}
            \frac{d\xi}{d\tau} &= \xi \left[ (\xi - v)^2 - c_s^2 (1 - \xi v)^2 \right], \\
            \frac{dv}{d\tau} &= 2 v c_s^2 (1 - v^2) (1 - \xi v), \\
            \frac{dw}{d\tau} &= w \left( 1 + \frac{1}{c_s^2} \right) \gamma^2 \mu \frac{dv}{d\tau}.
        \end{align}
    \end{itemize}
\end{frame}

\begin{frame}
    \frametitle{Fluid shells}
    \begin{itemize}
        \item Three types of solutions, determined by
        \begin{itemize}
            \item Wall velocity $v_\text{wall}$
            \item Transition strength $\alpha_n$
            \item Speed of sound $c_s(T,\phi)$
        \end{itemize}
    \end{itemize}
    \includegraphics[width=0.8\textwidth]{../fig/all_circle.pdf} \\
    {\footnotesize Black = bubble wall / phase boundary}
\end{frame}

\section{Equations of state}

\begin{frame}
    \frametitle{General equation of state}
    \begin{itemize}
        \item Energy-momentum tensor in momentum space
        \begin{equation}
            T^{\mu \nu}(x) = \int \frac{d^3 p}{(2 \pi)^3} \frac{p^\mu p^\nu}{p^0} f(\vec{p},x).
        \end{equation}
        \item Pressure
        \begin{align}
            p = \frac{1}{3} T^i_i
            &= \frac{1}{3} \int \frac{d^3 p}{(2 \pi)^3} \frac{|\vec{p}|^2}{E} f(\vec{p},x) \quad \big| \ f(\vec{p}) = \frac{1}{e^\frac{E-\mu}{T} - 1}, \ \mu = 0\\
            &= \frac{1}{6 \pi^2} \int_0^\infty \frac{p^3 dp}{e^\frac{p}{T} - 1}
            = \frac{1}{6 \pi^2} \Gamma(4) \zeta(4) T^2 \\
            &= \frac{\pi^2}{90} T^4
        \end{align}
        \item Accounting for multiple fields and an external potential
        \begin{align}
            p(T,\phi) = \frac{\pi^2}{90} g_p(T,\phi) T^4 - V(T,\phi)
        \end{align}
    \end{itemize}
\end{frame}

\begin{frame}
    \frametitle{Bag model: the simplest model}
    \begin{itemize}
        \item Equation of state with constant degrees of freedom
        \begin{equation}
            g_\pm = \frac{90}{\pi^2} a_\pm \quad \rightarrow \quad p_\pm = a_\pm T^4 - V_\pm
        \end{equation}
        \item The rest can be deduced with thermodynamics \\
        \begin{minipage}[t]{0.33\linewidth}
            \begin{itemize}
                \item Enthalpy density $w$
                \item Energy density $e$
                \item Entropy density $s$
                \item Sound speed $c_s$
            \end{itemize}%
        \end{minipage}%
        \hfill%
        \begin{minipage}[t]{0.33\linewidth}%
            \begin{align*}
                w
                &\equiv \frac{dH}{dV} \\
                &= e+p \\
                &= T \frac{\partial p}{\partial T} \\
                &= Ts
            \end{align*}
        \end{minipage}
        \hfill%
        \begin{minipage}[t]{0.33\linewidth}
            \begin{equation}
                c_s^2 \equiv \left( \frac{\partial p}{\partial e} \right)_s = \frac{1}{3}
            \end{equation}
        \end{minipage}
    \end{itemize}
\end{frame}

\begin{frame}
    \frametitle{Beyond the bag model}
    \begin{itemize}
        \item Non-constant degrees of freedom
            \item Varying sound speed $c_s(T,\phi)$
        \item Next approximation: constant sound speed model
        \begin{align}
            g_{p\pm} &= \frac{90}{\pi^2} a_\pm \left( \frac{T}{T_0} \right)^{\mu_\pm - 4} \\
            \Rightarrow \quad
            p_\pm &= a_\pm \left( \frac{T}{T_0} \right)^{\mu_\pm - 4} T^4 - V_\pm {\color{gray} \approx a_\pm T^{\mu_\pm} - V_\pm }
        \end{align}
        \item Constant speeds of sound
        \begin{equation}
            c_{s\pm}^2 = \frac{1}{\mu_\pm - 1}
        \end{equation}
    \end{itemize}
\end{frame}

\begin{frame}
    \frametitle{Equation of state from an arbitrary particle physics model}
    \begin{itemize}
        \item The equation of state can be constructed from $V(T,\phi)$ and two of $g_e(T,\phi), \ g_p(T,\phi), \ g_s(T,\phi)$
        \begin{align}
            g_p &= 4g_s - 3g_e \\
            e(T,\phi) &= \frac{\pi^2}{30} g_e(T,\phi) T^4 + V(T,\phi) \\
            p(T,\phi) &= \frac{\pi^2}{90} g_p(T,\phi) T^4 - V(T,\phi) \\
            s(T,\phi) &= \frac{2\pi^2}{45} g_s(T,\phi) T^3 \\
        \end{align}
    \end{itemize}
\end{frame}

\section{Simulation with PTtools}

\begin{frame}
    \frametitle{PTtools}
    \begin{itemize}
        \item Input
        \begin{itemize}
            \item Equation of state, either directly as $p(T,\phi)$ (and $e(T,\phi)$ or $s(T,\phi)$),
                or as degrees of freedom $g_s(T,\phi), g_e(T,\phi)$
            \item Bubble wall speed $v_\text{wall}$
            \item Transition strength parameter $\alpha_n$
            \item Transition rate parameter $\beta$
        \end{itemize}
        \item Output: gravitational wave power spectrum
        \item Use case: simulate GW spectra of various models to be compared with future LISA data in the 2030s
    \end{itemize}
\end{frame}

\begin{frame}
    \frametitle{Fluid shell algorithm}
    \begin{minipage}[t]{0.48\linewidth}%
        \begin{itemize}
            \item Detonations
            \begin{itemize}
                \item Solve boundary conditions at the wall, integrate from the wall to $v=0$
            \end{itemize}
            \item Deflagrations \& hybrids
            \begin{itemize}
                \item Guess an enthalpy behind the wall
                \item Solve boundary conditions at the wall
                \item Integrate to the shock
                \item Solve boundary conditions at the shock
                \item Check if $w=w_n$
                \item If not, change enthalpy guess
                \item If hybrid, integrate from the wall to $v=0$
            \end{itemize}%
        \end{itemize}%
    \end{minipage}%
    \hfill%
    \begin{minipage}[t]{0.48\linewidth}%
        % \includegraphics[width=0.2\textwidth]{../fig/shell_plot_vw_07_alphan_01_review.pdf}%
        TODO figure here
    \end{minipage}
\end{frame}

\begin{frame}
    \frametitle{Example result: entropy conservation}
    \begin{minipage}[t]{0.48\linewidth}
        \begin{itemize}
            \item 10 000 bubbles with varying $v_\text{wall}, \alpha_n$
            \item Top-left region ruled out by $\alpha_+ > \frac{1}{3}$
            \item Unphysical region (blue): decrease in total entropy
        \end{itemize}%
    \end{minipage}%
    \hfill%
    \begin{minipage}[t]{0.48\linewidth}%
        TODO figure here
    \end{minipage}
\end{frame}

\begin{frame}
    \frametitle{Gravitational wave production}
    \begin{itemize}
        \item Sound Shell Model (Hindmarsh, 2018)
        \item Velocity profile of a single bubble $v_\text{ip}(\xi)$ + sine transform \\ \textrightarrow \ velocity power spectrum
        \begin{align}
            \lambda(x) &= \frac{e(x) - \bar{e}}{\bar{w}} \\
            f(z) &= \frac{4 \pi}{z} \int_0^\infty d\xi v_\text{ip}(\xi) \sin(z\xi) \\
            l(z) &= \frac{4 \pi}{z} \int_0^\infty d\xi \lambda_\text{ip}(\xi) \xi \sin(z\xi) \\
            |A(z)|^2 &= \frac{1}{4} \left[ (f'(z))^2 + (c_s l(z))^2 \right]
        \end{align}
        \item Convolution with the nucleation rate function \\
            \textrightarrow Overall velocity power spectrum
        \item Convolution with a Green's function \textrightarrow \ GW power spectrum
        \begin{equation}
            \mathcal{P}_\text{gw} = \frac{1}{12 H^2} \frac{k^3}{2\pi^2} P_{\dot{h}}
        \end{equation}
    \end{itemize}
\end{frame}

\section{Conclusion}

\begin{frame}
    \frametitle{Summary}
    \begin{itemize}
        \item The hydrodynamics is based on energy-momentum conservation
        \item Solving a bubble
        \item PTtools will be published soon with support for arbitrary equations of state
    \end{itemize}
\end{frame}

\begin{frame}
    \frametitle{Sources}
    \begin{itemize}
        \scriptsize
        \item M. Hindmarsh, M. Lüben, J. Lumma, and M. Pauly, “Phase transitions in the early universe,” SciPost Phys. Lect. Notes, Feb. 2021, doi: 10.21468/SciPostPhysLectNotes.24.
        \item Hindmarsh, Mark, and Mulham Hijazi. “Gravitational Waves from First Order Cosmological Phase Transitions in the Sound Shell Model.” Journal of Cosmology and Astroparticle Physics 2019, Dec. 2019, doi: 10.1088/1475-7516/2019/12/062.
        \item Caprini et al. “Detecting Gravitational Waves from Cosmological Phase Transitions with LISA: An Update.” Journal of Cosmology and Astroparticle Physics 2020, Mar. 2020, doi: 10.1088/1475-7516/2020/03/024.
        \item Espinosa et al. “Energy Budget of Cosmological First-Order Phase Transitions.” Journal of Cosmology and Astroparticle Physics 2010, Jun. 2010, doi: 10.1088/1475-7516/2010/06/028.
        \item Giese et al., “Model-Independent Energy Budget for LISA.” Journal of Cosmology and Astroparticle Physics, Jan. 2021, doi: 10.1088/1475-7516/2021/01/072.
        \item Borsanyi, Sz, Z. Fodor, K. H. Kampert, S. D. Katz, T. Kawanai, T. G. Kovacs, S. W. Mages, et al. “Lattice QCD for Cosmology.”, Jun. 2016, ArXiv: 1606.07494
    \end{itemize}
\end{frame}

\begin{frame}
    % \frametitle{Thank you!}
    Thank you!
\end{frame}

\section{Extra slides}

% Extra backup slides
\begin{frame}
    \frametitle{Speedup example with Numba}
    \begin{itemize}
        \item Add JIT decorator
        \item Replace unsupported features with simpler code or split the unsupported parts to another function
        \item (Restructure the function to make the possible parallelism explicit)
    \end{itemize}
    TODO figure here
\end{frame}
